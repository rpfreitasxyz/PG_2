\documentclass[grad,numbers]{coppe}
\usepackage[utf8]{inputenc}
\usepackage{amsmath,amssymb}
\usepackage{hyperref}

\makelosymbols
\makeloabbreviations



  \RequirePackage[english, brazil]{babel}


\usepackage{color}
\usepackage{fancyvrb}
\newcommand{\VerbBar}{|}
\newcommand{\VERB}{\Verb[commandchars=\\\{\}]}
\DefineVerbatimEnvironment{Highlighting}{Verbatim}{commandchars=\\\{\}}
% Add ',fontsize=\small' for more characters per line
\usepackage{framed}
\definecolor{shadecolor}{RGB}{248,248,248}
\newenvironment{Shaded}{\begin{snugshade}}{\end{snugshade}}
\newcommand{\AlertTok}[1]{\textcolor[rgb]{0.94,0.16,0.16}{#1}}
\newcommand{\AnnotationTok}[1]{\textcolor[rgb]{0.56,0.35,0.01}{\textbf{\textit{#1}}}}
\newcommand{\AttributeTok}[1]{\textcolor[rgb]{0.77,0.63,0.00}{#1}}
\newcommand{\BaseNTok}[1]{\textcolor[rgb]{0.00,0.00,0.81}{#1}}
\newcommand{\BuiltInTok}[1]{#1}
\newcommand{\CharTok}[1]{\textcolor[rgb]{0.31,0.60,0.02}{#1}}
\newcommand{\CommentTok}[1]{\textcolor[rgb]{0.56,0.35,0.01}{\textit{#1}}}
\newcommand{\CommentVarTok}[1]{\textcolor[rgb]{0.56,0.35,0.01}{\textbf{\textit{#1}}}}
\newcommand{\ConstantTok}[1]{\textcolor[rgb]{0.00,0.00,0.00}{#1}}
\newcommand{\ControlFlowTok}[1]{\textcolor[rgb]{0.13,0.29,0.53}{\textbf{#1}}}
\newcommand{\DataTypeTok}[1]{\textcolor[rgb]{0.13,0.29,0.53}{#1}}
\newcommand{\DecValTok}[1]{\textcolor[rgb]{0.00,0.00,0.81}{#1}}
\newcommand{\DocumentationTok}[1]{\textcolor[rgb]{0.56,0.35,0.01}{\textbf{\textit{#1}}}}
\newcommand{\ErrorTok}[1]{\textcolor[rgb]{0.64,0.00,0.00}{\textbf{#1}}}
\newcommand{\ExtensionTok}[1]{#1}
\newcommand{\FloatTok}[1]{\textcolor[rgb]{0.00,0.00,0.81}{#1}}
\newcommand{\FunctionTok}[1]{\textcolor[rgb]{0.00,0.00,0.00}{#1}}
\newcommand{\ImportTok}[1]{#1}
\newcommand{\InformationTok}[1]{\textcolor[rgb]{0.56,0.35,0.01}{\textbf{\textit{#1}}}}
\newcommand{\KeywordTok}[1]{\textcolor[rgb]{0.13,0.29,0.53}{\textbf{#1}}}
\newcommand{\NormalTok}[1]{#1}
\newcommand{\OperatorTok}[1]{\textcolor[rgb]{0.81,0.36,0.00}{\textbf{#1}}}
\newcommand{\OtherTok}[1]{\textcolor[rgb]{0.56,0.35,0.01}{#1}}
\newcommand{\PreprocessorTok}[1]{\textcolor[rgb]{0.56,0.35,0.01}{\textit{#1}}}
\newcommand{\RegionMarkerTok}[1]{#1}
\newcommand{\SpecialCharTok}[1]{\textcolor[rgb]{0.00,0.00,0.00}{#1}}
\newcommand{\SpecialStringTok}[1]{\textcolor[rgb]{0.31,0.60,0.02}{#1}}
\newcommand{\StringTok}[1]{\textcolor[rgb]{0.31,0.60,0.02}{#1}}
\newcommand{\VariableTok}[1]{\textcolor[rgb]{0.00,0.00,0.00}{#1}}
\newcommand{\VerbatimStringTok}[1]{\textcolor[rgb]{0.31,0.60,0.02}{#1}}
\newcommand{\WarningTok}[1]{\textcolor[rgb]{0.56,0.35,0.01}{\textbf{\textit{#1}}}}
\providecommand{\tightlist}{%
  \setlength{\itemsep}{0pt}\setlength{\parskip}{0pt}}
\usepackage{longtable}
\usepackage{booktabs}
\begin{document}
  \title{\emph{Valuation} Intrínseco e Relativo: O estudo de caso da COPEL}
  \foreigntitle{Intrinsic and Relative Valuation: The case study of COPEL}
    \author{Rafael Pinto}{de Freitas}
  
    \advisor{Prof.}{José Roberto}{Ribas}{D.Sc.}
    \advisor{Prof.}{Nome do Segundo Orientador}{Sobrenome}{Ph.D}
  

    \examiner{Prof.}{José Roberto Ribas}{D.Sc.}
    \examiner{Prof.}{Nome Completo do Segundo Examinador}{Ph.D}
    \examiner{Prof.}{Nome Completo do Terceiro Examinador}{Ph.D}
  
  \department{EPR}
  \date{08}{2020}

    \keyword{Valuation}
    \keyword{Análise de investimentos}
  
  \maketitle

  \frontmatter
  \dedication{\begin{quote}
Judge a man by his questions rather than by his answers.

\hfill --- Voltaire
\end{quote}}
    \chapter*{Agradecimentos}
  Agradeço pela oportunidade de cursar um ensino superior de qualidade de forma pública. Mesmo com suas diversas limitações e imperfeições, a República brasileira segue em frente com a mensagem de democratização do conhecimento. É somente por meio desta que podemos nos defender contra a tirania vil da ignorância. Dessa forma, estou em dívida com a sociedade; com todos que permitiram minha entrada e estadia no curso de Engenharia de Produção pela UFRJ. Uma dívida monumental, se pensada pela ótica dos benefícios. Espero retornar o investimento em breve, a começar de forma humilde com este trabalho de conclusão de curso. Boa leitura!
  \begin{abstract}
Sit urna lacus aenean euismod morbi integer mauris ligula euismod. Massa leo nunc rutrum non vulputate viverra erat aliquet torquent. Dictumst inceptos litora diam dui eu non sodales eget metus? Mollis faucibus justo class class nulla vestibulum consequat purus.

Sit est ligula massa massa. Lectus parturient vehicula luctus nisl facilisis iaculis sagittis euismod ornare ut platea! Vestibulum et cras nostra luctus morbi cubilia et ante ornare luctus commodo facilisis nam. Lobortis ligula dictum tortor facilisis ante gravida habitasse cras laoreet. Vehicula pharetra vulputate non magna ut interdum habitant quam et class elementum arcu!

Adipiscing nulla laoreet magna dignissim nostra phasellus lacinia elementum est id! Rutrum arcu aliquet torquent porttitor ligula eget dictumst aenean. Lacus dictumst phasellus sed lobortis leo convallis velit mi imperdiet. Ultricies convallis id vestibulum morbi rutrum tortor diam volutpat euismod montes enim cras eros luctus duis rutrum integer.

Consectetur platea augue vitae vitae integer ad tincidunt torquent ac. Pharetra malesuada odio non lobortis dis aliquet arcu nascetur magna porttitor. Lacinia curabitur primis ligula magna sociosqu hendrerit sociosqu risus cubilia. Arcu potenti mi pellentesque nulla per varius vitae lectus pellentesque! Tempor.
  \end{abstract}
  \begin{foreignabstract}
Sit urna lacus aenean euismod morbi integer mauris ligula euismod. Massa leo nunc rutrum non vulputate viverra erat aliquet torquent. Dictumst inceptos litora diam dui eu non sodales eget metus? Mollis faucibus justo class class nulla vestibulum consequat purus.

Sit est ligula massa massa. Lectus parturient vehicula luctus nisl facilisis iaculis sagittis euismod ornare ut platea! Vestibulum et cras nostra luctus morbi cubilia et ante ornare luctus commodo facilisis nam. Lobortis ligula dictum tortor facilisis ante gravida habitasse cras laoreet. Vehicula pharetra vulputate non magna ut interdum habitant quam et class elementum arcu!

Adipiscing nulla laoreet magna dignissim nostra phasellus lacinia elementum est id! Rutrum arcu aliquet torquent porttitor ligula eget dictumst aenean. Lacus dictumst phasellus sed lobortis leo convallis velit mi imperdiet. Ultricies convallis id vestibulum morbi rutrum tortor diam volutpat euismod montes enim cras eros luctus duis rutrum integer.

Consectetur platea augue vitae vitae integer ad tincidunt torquent ac. Pharetra malesuada odio non lobortis dis aliquet arcu nascetur magna porttitor. Lacinia curabitur primis ligula magna sociosqu hendrerit sociosqu risus cubilia. Arcu potenti mi pellentesque nulla per varius vitae lectus pellentesque! Tempor.
  \end{foreignabstract}
  \tableofcontents

  \listoffigures

  \listoftables

  \printlosymbols
  \printloabbreviations

  \mainmatter

  \hypertarget{introduuxe7uxe3o}{%
  \chapter{Introdução}\label{introduuxe7uxe3o}}
  
  Placeholder
  
  \hypertarget{contextualizauxe7uxe3o}{%
  \section{Contextualização}\label{contextualizauxe7uxe3o}}
  
  \hypertarget{justificativa}{%
  \section{Justificativa}\label{justificativa}}
  
  \hypertarget{objetivos}{%
  \section{Objetivos}\label{objetivos}}
  
  \hypertarget{limitauxe7uxf5es}{%
  \section{Limitações}\label{limitauxe7uxf5es}}
  
  \hypertarget{estrutura-do-trabalho}{%
  \section{Estrutura do trabalho}\label{estrutura-do-trabalho}}
  
  \hypertarget{o-mercado-de-energia}{%
  \chapter{O mercado de energia}\label{o-mercado-de-energia}}
  
  Placeholder
  
  \hypertarget{uxf3rguxe3os-presentes-no-estudo}{%
  \section{Órgãos presentes no estudo}\label{uxf3rguxe3os-presentes-no-estudo}}
  
  \hypertarget{mme}{%
  \subsection{MME}\label{mme}}
  
  \hypertarget{aneel}{%
  \subsection{ANEEL}\label{aneel}}
  
  \hypertarget{ons}{%
  \subsection{ONS}\label{ons}}
  
  \hypertarget{ccee}{%
  \subsection{CCEE}\label{ccee}}
  
  \hypertarget{epe}{%
  \subsection{EPE}\label{epe}}
  
  \hypertarget{o-fluxo-de-energia}{%
  \section{O fluxo de energia}\label{o-fluxo-de-energia}}
  
  \hypertarget{estudos-e-projeuxe7uxf5es-de-longo-prazo}{%
  \section{Estudos e projeções de longo prazo}\label{estudos-e-projeuxe7uxf5es-de-longo-prazo}}
  
  \hypertarget{plano-nacional-de-energia-pne}{%
  \subsection{Plano Nacional de Energia (PNE)}\label{plano-nacional-de-energia-pne}}
  
  \hypertarget{plano-decenal-de-expansuxe3o-de-energia-pde}{%
  \subsection{Plano Decenal de Expansão de Energia (PDE)}\label{plano-decenal-de-expansuxe3o-de-energia-pde}}
  
  \hypertarget{referencial-teuxf3rico}{%
  \chapter{Referencial teórico}\label{referencial-teuxf3rico}}
  
  Nesta seção será feita uma consideração a respeito dos métodos e conceitos utilizados ao longo do estudo. É de interesse do leitor prestar especial atenção ao enunciado abaixo, uma vez que é um breve alicerce teórico que serve não apenas para esse estudo, como para diversos outros similares.
  
  \hypertarget{demonstrauxe7uxf5es-financeiras}{%
  \section{Demonstrações financeiras}\label{demonstrauxe7uxf5es-financeiras}}
  
  Demonstrações financeiras são representações estruturadas da posição financeira e do desempenho financeiro de uma determinada entidade. Em uma analogia com o ser humano, o médico estuda os resultados de exames com a finalidade de averiguar a saúde de um indivíduo, sejam eles de sangue, tomografias, radiografias, dentre outros; o investidor astuto estuda as demonstrações financeiras da empresa, desde seu balanço patrimonial ao seu demonstrativo de fluxo de caixa, para averiguar a saúde financeira da empresa em questão.
  
  Essas demonstrações são ``escritas'' com base em princípios geralmente aceitos de contabilidade. Esse conjunto, no Brasil, é formado por normas provenientes de diversos órgãos, como: (1) a lei 6.404/76; (2) as leis 11.638/07 e 11.941/09; (3) resoluções, circulares, comunicados e afins de órgãos como o Conselho Federal de Contabilidade, Banco Central, e Conselho de Valores Mobiliários.
  
  A seguir, serão comentadas sobre as três principais demonstrações financeiras utilizadas para análise de negócios. Para fins de melhor elucidação, serão utilizadas as demonstrações da própria COPEL.
  
  \hypertarget{demonstrativo-de-resultados-do-exercuxedcio-dre}{%
  \subsection{Demonstrativo de Resultados do Exercício (DRE)}\label{demonstrativo-de-resultados-do-exercuxedcio-dre}}
  \begin{Shaded}
  \begin{Highlighting}[]
  \KeywordTok{library}\NormalTok{(tidyverse)}
  \end{Highlighting}
  \end{Shaded}
  \begin{verbatim}
  -- Attaching packages ------------------------------------ tidyverse 1.3.0 --
  \end{verbatim}
  \begin{verbatim}
  v ggplot2 3.3.2     v purrr   0.3.4
  v tibble  3.0.3     v dplyr   1.0.1
  v tidyr   1.1.0     v stringr 1.4.0
  v readr   1.3.1     v forcats 0.5.0
  \end{verbatim}
  \begin{verbatim}
  -- Conflicts --------------------------------------- tidyverse_conflicts() --
  x dplyr::filter() masks stats::filter()
  x dplyr::lag()    masks stats::lag()
  \end{verbatim}
  \begin{Shaded}
  \begin{Highlighting}[]
  \NormalTok{dre <-}\StringTok{ }\KeywordTok{tribble}\NormalTok{(}\OperatorTok{~}\NormalTok{x, }\OperatorTok{~}\NormalTok{y,}
                 \DecValTok{1}\NormalTok{, }\DecValTok{2}\NormalTok{,}
                 \DecValTok{3}\NormalTok{, }\DecValTok{4}\NormalTok{)}
  
  \NormalTok{knitr}\OperatorTok{::}\KeywordTok{kable}\NormalTok{(dre, }\StringTok{"latex"}\NormalTok{, }\DataTypeTok{caption =} \StringTok{"Demonstração de Resultado COPEL"}\NormalTok{)}
  \end{Highlighting}
  \end{Shaded}
  \begin{table}
  
  \caption{\label{tab:unnamed-chunk-1}Demonstração de Resultado COPEL}
  \centering
  \begin{tabular}[t]{r|r}
  \hline
  x & y\\
  \hline
  1 & 2\\
  \hline
  3 & 4\\
  \hline
  \end{tabular}
  \end{table}
  \hypertarget{balanuxe7o-patrimonial-bp}{%
  \subsection{Balanço Patrimonial (BP)}\label{balanuxe7o-patrimonial-bp}}
  
  \hypertarget{demonstrativo-de-fluxo-de-caixa-dfc}{%
  \subsection{Demonstrativo de Fluxo de Caixa (DFC)}\label{demonstrativo-de-fluxo-de-caixa-dfc}}
  
  \hypertarget{valuation-intruxednseco}{%
  \section{\texorpdfstring{\emph{Valuation} intrínseco}{Valuation intrínseco}}\label{valuation-intruxednseco}}
  
  \hypertarget{muxe9todo-do-fluxo-de-caixa-descontado}{%
  \subsection{Método do Fluxo de Caixa Descontado}\label{muxe9todo-do-fluxo-de-caixa-descontado}}
  
  \hypertarget{valuation-relativo}{%
  \section{\texorpdfstring{\emph{Valuation} relativo}{Valuation relativo}}\label{valuation-relativo}}
  
  \hypertarget{anuxe1lise-por-muxfaltiplos}{%
  \subsection{Análise por múltiplos}\label{anuxe1lise-por-muxfaltiplos}}
  
  \hypertarget{estudo-de-caso}{%
  \chapter{Estudo de caso}\label{estudo-de-caso}}
  
  Placeholder
  
  \hypertarget{contextualizauxe7uxe3o-da-copel}{%
  \section{Contextualização da COPEL}\label{contextualizauxe7uxe3o-da-copel}}
  
  \hypertarget{histuxf3ria}{%
  \subsection{História}\label{histuxf3ria}}
  
  \hypertarget{core-business}{%
  \subsection{\texorpdfstring{\emph{Core business}}{Core business}}\label{core-business}}
  
  \hypertarget{gerauxe7uxe3o}{%
  \subsubsection{Geração}\label{gerauxe7uxe3o}}
  
  \hypertarget{transmissuxe3o}{%
  \subsubsection{Transmissão}\label{transmissuxe3o}}
  
  \hypertarget{distribuiuxe7uxe3o}{%
  \subsubsection{Distribuição}\label{distribuiuxe7uxe3o}}
  
  \hypertarget{outros}{%
  \subsubsection{Outros}\label{outros}}
  
  \hypertarget{cuxe1lculo-do-valuation-intruxednseco}{%
  \section{\texorpdfstring{Cálculo do \emph{valuation} intrínseco}{Cálculo do valuation intrínseco}}\label{cuxe1lculo-do-valuation-intruxednseco}}
  
  \hypertarget{o-custo-de-capital-muxe9dio-ponderado-wacc}{%
  \subsection{O custo de capital médio ponderado (WACC)}\label{o-custo-de-capital-muxe9dio-ponderado-wacc}}
  
  \hypertarget{custo-de-capital-pruxf3prio}{%
  \subsubsection{Custo de capital próprio}\label{custo-de-capital-pruxf3prio}}
  
  \hypertarget{custo-de-capital-de-terceiros}{%
  \subsubsection{Custo de capital de terceiros}\label{custo-de-capital-de-terceiros}}
  
  \hypertarget{fluxo-de-caixa-descontado}{%
  \subsubsection{Fluxo de caixa descontado}\label{fluxo-de-caixa-descontado}}
  
  \hypertarget{cuxe1lculo-do-valuation-relativo}{%
  \section{\texorpdfstring{Cálculo do \emph{valuation} relativo}{Cálculo do valuation relativo}}\label{cuxe1lculo-do-valuation-relativo}}
  
  \hypertarget{margem-bruta}{%
  \subsection{Margem bruta}\label{margem-bruta}}
  
  \hypertarget{lucros-antes-de-juros-e-impostos-ebit}{%
  \subsection{Lucros antes de juros e impostos (EBIT)}\label{lucros-antes-de-juros-e-impostos-ebit}}
  
  \hypertarget{margem-luxedquida}{%
  \subsection{Margem líquida}\label{margem-luxedquida}}
  
  \hypertarget{razuxe3o-preuxe7olucro-pe}{%
  \subsection{Razão preço/lucro (P/E)}\label{razuxe3o-preuxe7olucro-pe}}
  
  \hypertarget{retorno-sobre-patrimuxf4nio-luxedquido-roe}{%
  \subsection{Retorno sobre patrimônio líquido (ROE)}\label{retorno-sobre-patrimuxf4nio-luxedquido-roe}}
  
  \hypertarget{comparauxe7uxe3o-com-empresas-do-setor}{%
  \subsection{Comparação com empresas do setor}\label{comparauxe7uxe3o-com-empresas-do-setor}}
  
  \hypertarget{conclusuxe3o}{%
  \chapter{Conclusão}\label{conclusuxe3o}}
  
  \hypertarget{referuxeancias-bibliogruxe1ficas}{%
  \chapter*{Referências Bibliográficas}\label{referuxeancias-bibliogruxe1ficas}}
  \addcontentsline{toc}{chapter}{Referências Bibliográficas}
  
  Placeholder

  \backmatter
  \bibliographystyle{coppe-unsrt}
  \bibliography{thesis}

  %\appendix
  %\include{appenA}
\end{document}

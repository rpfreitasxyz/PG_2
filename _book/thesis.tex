\documentclass[grad,numbers]{coppe}
\usepackage[utf8]{inputenc}
\usepackage{amsmath,amssymb}
\usepackage{hyperref}

\makelosymbols
\makeloabbreviations



  \RequirePackage[english, brazil]{babel}


\newlength{\cslhangindent}
\setlength{\cslhangindent}{1.5em}
\newenvironment{cslreferences}%
  {}%
  {\par}

\providecommand{\tightlist}{%
  \setlength{\itemsep}{0pt}\setlength{\parskip}{0pt}}
\usepackage{longtable}
\usepackage{booktabs}
\begin{document}
  \title{\emph{Valuation} Intrínseco e Relativo: O estudo de caso da COPEL}
  \foreigntitle{Intrinsic and Relative Valuation: The case study of COPEL}
    \author{Rafael Pinto}{de Freitas}
  
    \advisor{Prof.}{José Roberto}{Ribas}{D.Sc.}
    \advisor{Prof.}{Marco Ludwik Patrício}{Krebs}{D.Sc.}
  

    \examiner{Prof.}{José Roberto Ribas}{D.Sc.}
    \examiner{Prof.}{Marco Ludwik Patrício Krebs}{D.Sc.}
    \examiner{Prof.}{Nome Completo do Terceiro Examinador}{Ph.D}
  
  \department{EPR}
  \date{11}{2020}

    \keyword{Valuation}
    \keyword{Análise de investimentos}
  
  \maketitle

  \frontmatter
  \dedication{\begin{quote}
Judge a man by his questions rather than by his answers.

\hfill --- Voltaire
\end{quote}}
    \chapter*{Agradecimentos}
  Agradeço pela oportunidade de cursar um ensino superior de qualidade de forma pública. Mesmo com suas diversas limitações e imperfeições, a República brasileira segue em frente com a mensagem de democratização do conhecimento. É somente por meio desta que podemos nos defender contra a tirania vil da ignorância. Dessa forma, estou em dívida com a sociedade; com todos que permitiram minha entrada e estadia no curso de Engenharia de Produção pela UFRJ. Uma dívida monumental, se pensada pela ótica dos benefícios. Espero retornar o investimento em breve, a começar de forma humilde com este trabalho de conclusão de curso. Boa leitura!

  Lorem ipsum
  \begin{abstract}
Sit urna lacus aenean euismod morbi integer mauris ligula euismod. Massa leo nunc rutrum non vulputate viverra erat aliquet torquent. Dictumst inceptos litora diam dui eu non sodales eget metus? Mollis faucibus justo class class nulla vestibulum consequat purus.

Sit est ligula massa massa. Lectus parturient vehicula luctus nisl facilisis iaculis sagittis euismod ornare ut platea! Vestibulum et cras nostra luctus morbi cubilia et ante ornare luctus commodo facilisis nam. Lobortis ligula dictum tortor facilisis ante gravida habitasse cras laoreet. Vehicula pharetra vulputate non magna ut interdum habitant quam et class elementum arcu!

Adipiscing nulla laoreet magna dignissim nostra phasellus lacinia elementum est id! Rutrum arcu aliquet torquent porttitor ligula eget dictumst aenean. Lacus dictumst phasellus sed lobortis leo convallis velit mi imperdiet. Ultricies convallis id vestibulum morbi rutrum tortor diam volutpat euismod montes enim cras eros luctus duis rutrum integer.

Consectetur platea augue vitae vitae integer ad tincidunt torquent ac. Pharetra malesuada odio non lobortis dis aliquet arcu nascetur magna porttitor. Lacinia curabitur primis ligula magna sociosqu hendrerit sociosqu risus cubilia. Arcu potenti mi pellentesque nulla per varius vitae lectus pellentesque! Tempor.
  \end{abstract}
  \begin{foreignabstract}
Sit urna lacus aenean euismod morbi integer mauris ligula euismod. Massa leo nunc rutrum non vulputate viverra erat aliquet torquent. Dictumst inceptos litora diam dui eu non sodales eget metus? Mollis faucibus justo class class nulla vestibulum consequat purus.

Sit est ligula massa massa. Lectus parturient vehicula luctus nisl facilisis iaculis sagittis euismod ornare ut platea! Vestibulum et cras nostra luctus morbi cubilia et ante ornare luctus commodo facilisis nam. Lobortis ligula dictum tortor facilisis ante gravida habitasse cras laoreet. Vehicula pharetra vulputate non magna ut interdum habitant quam et class elementum arcu!

Adipiscing nulla laoreet magna dignissim nostra phasellus lacinia elementum est id! Rutrum arcu aliquet torquent porttitor ligula eget dictumst aenean. Lacus dictumst phasellus sed lobortis leo convallis velit mi imperdiet. Ultricies convallis id vestibulum morbi rutrum tortor diam volutpat euismod montes enim cras eros luctus duis rutrum integer.

Consectetur platea augue vitae vitae integer ad tincidunt torquent ac. Pharetra malesuada odio non lobortis dis aliquet arcu nascetur magna porttitor. Lacinia curabitur primis ligula magna sociosqu hendrerit sociosqu risus cubilia. Arcu potenti mi pellentesque nulla per varius vitae lectus pellentesque! Tempor.
  \end{foreignabstract}
  \tableofcontents

  \listoffigures

  \listoftables

  \printlosymbols
  \printloabbreviations

  \mainmatter

  \hypertarget{introduuxe7uxe3o}{%
  \chapter{Introdução}\label{introduuxe7uxe3o}}

  Placeholder

  \hypertarget{contextualizauxe7uxe3o}{%
  \section{Contextualização}\label{contextualizauxe7uxe3o}}

  \hypertarget{justificativa}{%
  \section{Justificativa}\label{justificativa}}

  \hypertarget{objetivos}{%
  \section{Objetivos}\label{objetivos}}

  \hypertarget{delimitauxe7uxf5es}{%
  \section{Delimitações}\label{delimitauxe7uxf5es}}

  \hypertarget{estrutura-do-trabalho}{%
  \section{Estrutura do trabalho}\label{estrutura-do-trabalho}}

  \hypertarget{o-mercado-de-energia}{%
  \chapter{O mercado de energia}\label{o-mercado-de-energia}}

  Placeholder

  \hypertarget{uxf3rguxe3os-presentes-no-estudo}{%
  \section{Órgãos presentes no estudo}\label{uxf3rguxe3os-presentes-no-estudo}}

  \hypertarget{mme}{%
  \subsection{MME}\label{mme}}

  \hypertarget{aneel}{%
  \subsection{ANEEL}\label{aneel}}

  \hypertarget{ons}{%
  \subsection{ONS}\label{ons}}

  \hypertarget{ccee}{%
  \subsection{CCEE}\label{ccee}}

  \hypertarget{epe}{%
  \subsection{EPE}\label{epe}}

  \hypertarget{o-fluxo-de-energia}{%
  \section{O fluxo de energia}\label{o-fluxo-de-energia}}

  \hypertarget{estudos-e-projeuxe7uxf5es-de-longo-prazo}{%
  \section{Estudos e projeções de longo prazo}\label{estudos-e-projeuxe7uxf5es-de-longo-prazo}}

  \hypertarget{plano-nacional-de-energia-pne}{%
  \subsection{Plano Nacional de Energia (PNE)}\label{plano-nacional-de-energia-pne}}

  \hypertarget{plano-decenal-de-expansuxe3o-de-energia-pde}{%
  \subsection{Plano Decenal de Expansão de Energia (PDE)}\label{plano-decenal-de-expansuxe3o-de-energia-pde}}

  \hypertarget{referencial-teuxf3rico}{%
  \chapter{Referencial teórico}\label{referencial-teuxf3rico}}

  Nesta seção será feita uma consideração a respeito dos métodos e conceitos utilizados ao longo do estudo. É de interesse do leitor prestar especial atenção ao enunciado abaixo, uma vez que é um breve alicerce teórico que serve não apenas para esse estudo, como para diversos outros similares. Recomenda-se, ainda, a leitura de Damodaran (\protect\hyperlink{ref-damodaran2007}{2007}) para se ter um panorama de diversos outros modelos a serem aplicados, assim como uma discussão a respeito de seus usos e efetividade.

  \hypertarget{valuation-intruxednseco}{%
  \section{\texorpdfstring{\emph{Valuation} intrínseco}{Valuation intrínseco}}\label{valuation-intruxednseco}}

  Comecemos discutindo brevemente a respeito do que é possuir ``valor intrínseco''. Este é um conceito filosófico, em que o valor de um objeto ou projeto é derivado de, e por si só -- em outras palavras, livre de fatores externos. Analistas financeiros constroem modelos para estimar o que se imagina ser o valor intrínseco de uma empresa sem considerar o seu valor de mercado em determinado dia.

  Naturalmente, o mercado, no curto prazo, está sujeito a flutuações que podem ser atribuídas a diversos fatores, desde manipulação de preços em papéis mais ilíquidos a pensamento de manada por parte de investidores. Cabe, nesse momento, utilizar a analogia popularizada por Graham (\protect\hyperlink{ref-graham2016}{2016}, pp.~204-205), do Sr.~Mercado:
  \begin{quote}
  ``Imagine que você possui uma participação pequena em uma companhia de capital fechado que lhe custou US\$1.000. Um de seus sócios, chamado Sr.~Mercado, é de fato muito prestativo. Todo dia ele lhe informa o que pensa ser o valor de sua participação e, além disso, se dispõe a comprar de você ou vender a você uma participação adicional naquelas bases. Às vezes, sua ideia de valor parece plausível e justificada pela evolução e pelas perspectivas do negócio da forma como você as conhece. Por outro lado, o Sr.~Mercado deixa frequentemente o entusiasmo ou o receio tomar conta dele e o valor proposto por ele lhe parece pura bobagem. Se você é um investidor prudente ou um empresário inteligente, deixaria as comunicações diárias do Sr.~Mercado influenciarem sua opinião sobre o valor de uma participação de US\$1.000 na companhia? Só se você concordasse com ele ou então desejasse negociar com ele. Você pode ficar feliz em vender para ele quando ele cota um preço ridiculamente alto e igualmente feliz em comprar dele quando seu preço é baixo. No entanto, no resto do tempo, você seria mais esperto se formulasse suas próprias ideias acerca do valor de sua carteira com base nos relatório completos da companhia sobre suas operações e posições financeiras.''
  \end{quote}
  Dessa forma, a discrepância entre preço de mercado e a estimativa do valor intrísenco feita por um analista torna-se uma medida para oportunidade de investimento. Aqueles que considerarem tais modelos como estimativas razoáveis de valor intrínseco, e que tomarem ação baseando-se nessas estimativas, são conhecidos como investidores de valor (DAMODARAN, \protect\hyperlink{ref-damodaran2012}{2012}).

  \hypertarget{anuxe1lise-de-fluxo-de-caixa-descontado}{%
  \subsection{Análise de Fluxo de Caixa Descontado}\label{anuxe1lise-de-fluxo-de-caixa-descontado}}

  A análise de fluxos de caixa descontados (DCF) é um método de se descobrir o valor de uma ação, projeto, empresa, ou ativo usando os conceitos do valor temporal do dinheiro. Para se aplicar o método, todos os fluxos de caixa futuros são estimados e descontados ao se utilizar o custo de capital para dar seus valores presentes. A soma de todos os futuros fluxos de caixa, tanto de entrada quanto de saída, resulta no valor presente líquido, que é tomado como o valor dos fluxos de caixa em questão, no momento.

  Seguindo a queda do mercado em 1929, o método ganhou popularidade para se avaliar ações. De fato, provavelmente o primeiro a formalizar a expressão do método em termos econômicos modernos foi Fisher (\protect\hyperlink{ref-fisher1930}{1930}).

  O valor presente líquido pode ser expresso matematicamente como: \[
  VPL = \sum_{i=0}^N\frac{FC_t}{(1+r)^t}
  \] onde \(FC_t\) é o fluxo de caixa no tempo \(t\) e \(r\) é a taxa de desconto. Naturalmente, para que o somatório acima esteja correto, assume-se que a taxa de desconto permaneça constante através do período todo. Caso assuma-se que o fluxo de caixa continue indefinidamente, a previsão finita é geralmente combinada com a premissa de um crescimento constante de fluxo de caixa além do período de projeção discreto -- a dita perpetuidade. Matematicamente: \[
  VPL = \sum_{i=0}^N\frac{FC_t}{(1+r)^t} + \frac{FC_{N+2}}{(1+r)^{N+1}(r-g)}
  \] onde o somatório é o período de crescimento normal, e além do somatório, temos o fluxo de caixa em perpetuidade, sendo descontado.

  A pergunta, entretanto, repousa sobre encontrar a taxa de desconto. Diversos modelos foram apresentados com tal finalidade, sendo o mais utilizado o \emph{capital asset pricing model}, doravante mencionado pela sua abreviatura, CAPM.

  Este modelo foi introduzido por Sharpe (\protect\hyperlink{ref-sharpe1964}{1964}), desenvolvendo em cima do trabalho iniciado em diversificação e teoria moderna do portfólio (MARKOWITZ, \protect\hyperlink{ref-markowitz1952}{1952}). O CAPM leva em conta a sensibilidade de um ativo ao risco não diversificável -- também conhecido como risco sistemático, ou risco de mercado -- geralmente representado por \(\beta\). De fato, a equação é como segue: \[
  E(R_i) = R_f + \beta_i(E(R_m)-R_f)
  \] onde \(E(R_i)\) é o retorno esperado do ativo, \(R_f\) é a taxa de juros livre de risco -- oriunda geralmente de títulos do governo -- e \(E(R_m)\) é o retorno esperado do mercado. \(\beta\), como comentado, é a sensibilidade do ativo em relação ao mercado, de forma tal que: \[
  \beta_i = \frac{Cov(R_i, R_m)}{Var(R_m)} = \rho_{i,m}\frac{\sigma_i}{\sigma_m}
  \] onde \(\rho_{i,m}\) denota o coeficiente de correlação entre o investimento \(i\) e o mercado \(m\), \(\sigma_i\) é o desvio padrão para o investimento \(i\), e \(\sigma_m\) é o desvio padrão para o mercado \(m\).

  Podemos entender essa equação melhor se a rearranjarmos: \[
  E(R_i) = R_f + \beta_i(E(R_m)-R_f) \iff E(R_i) - R_f = \beta_i(E(R_m)-R_f)
  \] onde o segundo lado denota uma equivalência interessante. Dessa forma, podemos dizer que o prêmio de risco para o ativo individual é igual ao prêmio pelo risco de mercado, multiplicado pela sensibilidade do ativo (\(\beta\)).

  De fato, o modelo leva em conta diversas premissas, dentre as quais todos os investidores (ARNOLD, \protect\hyperlink{ref-arnold2008}{2008}):
  \begin{enumerate}
  \def\labelenumi{\arabic{enumi}.}
  \tightlist
  \item
    Têm por objetivo maximizar utilidades econômicas (quantidades de ativos são dadas e fixas).
  \item
    São racionais e aversos a risco.
  \item
    São amplamente diversificados sobre uma grande gama de investimentos.
  \item
    São tomadores de preço, isto é, não influenciam nos preços.
  \item
    Podem emprestar e tomar quantias ilimitadas sob a taxa livre de risco de juros.
  \item
    Fazem trocas sem custo de transação ou impostos.
  \item
    Lidam com ativos que são todos altamente diversificáveis em pequenas parcelas -- são perfeitamente divisíveis e líquidos).
  \item
    Têm expectativas homogêneas.
  \item
    Têm todas as informações disponíveis ao mesmo tempo.
  \end{enumerate}
  Naturalmente, pela quantidade e teor das premissas, este é um modelo que fortemente simplifica a realidade. De fato, pela sua lógica simples e fácil aplicabilidade, ainda é muito utilizado na indústria, embora a maior parte das aplicações utilizando-se este modelo sejam consideradas inválidas (FAMA; FRENCH, \protect\hyperlink{ref-fama2004}{2004}).

  Para os propósitos deste trabalho, entretanto, será feita uma análise de fluxos de caixa descontados utilizando-se, também, o CAPM.

  \hypertarget{modelo-de-desconto-de-dividendos}{%
  \subsection{Modelo de Desconto de Dividendos}\label{modelo-de-desconto-de-dividendos}}

  O modelo de desconto de dividendos (DDM) é um método de se fazer o \emph{valuation} de uma ação baseado na teoria de que a ação vale a soma de todos os seus pagamentos de dividendos futuros, descontados de volta ao seu valor presente líquido (VPL). A equação mais utilizada amplamente é o chamado modelo de crescimento de Gordon (GGM). É nomeada assim por causa da publicação de Gordon e Shapiro (\protect\hyperlink{ref-gordon1959}{1959}), embora tenha sido originalmente desenvolvida três anos antes (GORDON; SHAPIRO, \protect\hyperlink{ref-gordon1956}{1956}). Trata-se da equação: \[
  P_0 = \frac{D_1}{r-g}
  \] onde \(P_0\) é o valor atual da ação, \(g\) é a taxa de crescimento constante em perpetuidade esperada dos dividendos, \(r\) é o custo de capital próprio da empresa; e \(D_1\) é o valor dos dividendos do próximo ano.

  Naturalmente, existem alguns pressupostos deste modelo:
  \begin{enumerate}
  \def\labelenumi{\roman{enumi}.}
  \tightlist
  \item
    Uma taxa de crescimento constante e perpétua, menor que o custo de capital.
  \item
    A ação deve pagar dividendos regularmente; do contrário, versões mais generalizadas do modelo de desconto de dividendos devem ser usados para se descobrir o valor da ação.
  \end{enumerate}
  A partir destes pontos, temos que as violações de (i) identifica uma ação de valor negativo; e (ii) provê um valor errôneo -- caso seja levado ao extremo, uma empresa que não paga dividendos efetivamente não valeria nada.

  A solução para (i) é se considerar um modelo de desconto de dividendos de dois estágios, isto é: \[
  P_0 = \frac{D_0 (1+g)}{r-g} \left[1 - \frac{(1+g)^N}{(1+r)^N}\right] + \frac{D_0 (1+g)^N (1+g_\infty)}{(1+r)^N (r-g_\infty)}
  \] onde \(D_0\) denota os dividendos deste ano, \(g\) a taxa de crescimento esperada de curto prazo, \(g_\infty\) a taxa de crescimento de longo prazo, e \(N\) o período (em número de anos), através do qual a taxa de curto prazo é aplicada.

  Uma solução comum para (ii) seria assumir que a hipótese de Modigliani-Miller de irrelevância de dividendos (MODIGLIANI; MILLER, \protect\hyperlink{ref-modigliani1958}{1958}) seja verdadeira, e então substituir os dividendos \(D\) por \(E\), os lucros por ação. Entretanto, isso requer o uso de crescimento dos lucros, ao invés dos de dividendos, que podem ser diferentes.

  A equação de Gordon pode ser entendida como o fato de que o retorno total de uma ação é igual à soma da sua receita e seus ganhos de capital. De fato, se rearranjada, teremos que: \[
  P_0 = \frac{D_1}{r-g} \iff \frac{D_1}{P_0} + g = r
  \] o que significa que o \emph{dividend yield} (\(D_1/P_0\)) mais o crescimento (\(g\)) é igual ao custo de capital próprio (\(r\)). Ora, caso consideremos a taxa de crescimento de dividendos no modelo como um \emph{proxy} para o crescimento de lucros e, por extensão, o preço da ação e os ganhos de capital. Consideraríamos, então, o custo de capital próprio como um \emph{proxy} para o retorno total requerida pelo investidor.

  \hypertarget{valuation-relativo}{%
  \section{\texorpdfstring{\emph{Valuation} relativo}{Valuation relativo}}\label{valuation-relativo}}

  Em \emph{valuation} relativo, um determinado ativo é avaliado baseado em quão precificados estão os ativos similares no mercado. Como exemplo, uma comprador de imóveis pode, antes de realizar uma compra à vista/financiamento abrupto, pode pesquisar por imóveis similares na vizinhança. Ora, uma pessoa que coleciona selos pode fazer um julgamento de quanto vai pagar em outro selo raro ao checar preços de transações desse mesmo selo em outras épocas. Dessa forma, um investidor em potencial pode estimar o preço de uma ação a comprar fazendo uma pesquisa através da precificação de ações ``similares''.

  Pela descrição acima, existem três fatores a se considerar:
  \begin{enumerate}
  \def\labelenumi{\arabic{enumi}.}
  \tightlist
  \item
    \textbf{É necessário encontrar ativos comparáveis, precificados pelo mercado.} Esta é uma tarefa que é mais fácil de se cumprir com ativos tangíveis do que com imateriais, como ações. Frequentemente, analistas consideram outras empresas do mesmo setor como comparáveis, comparando por exemplo, empresas de utilidade com outras empresas de utilidade.
  \item
    \textbf{É importante traduzir os preços de mercado a uma variável comum.} A finalidade disso é gerar preços padronizados que sejam comparáveis. Embora isso não seja necessário com ativos idênticos, é necessário quando existe heterogenia. Considere, por exemplo, o exemplo dos imóveis. Uma casa tem 200 m² e outra, 100 m². Reduziria-se um fator à metragem. Naturalmente, com empresas acontece algo similar, geralmente reduzindo-se a múltiplos de lucros, valor contábil, dentre outros.
  \item
    \textbf{É necessário ajustar-se diferenças entre ativos.} Novamente, consideremos o exemplo da casa. Ambas possuem a mesma metragem, mas uma acabou de ser construída, e outra tem mais de 200 anos de idade. Ora, havendo essa diferença de idades, \emph{ceteris paribus}, a casa mais nova deve valer mais. Com ações, pode haver algo similar. Empresas de maior crescimento, \emph{ceteris paribus}, devem valer mais do que empresas de menor crescimento, por exemplo.
  \end{enumerate}
  Cabe comentar que existe uma diferença filosófica significativa entre as abordagens intrínseca e relativa. Através de \emph{valuation} intrínseco, tentamos estimar o valor de um ativo baseado na sua capacidade de gerar fluxos de caixa no futuro. No \emph{valuation} relativo, estamos fazendo um julgamento em quanto um ativo vale ao olharmos para o que o mercado está pagando por ativos similares -- implicitamente, estamos ``confiando'' no julgamento de valor do mercado. Dessa forma, caso o mercado esteja sistematicamente superestimando ou subestimando -- \emph{bull} e \emph{bear market}, respectivamente -- um grupo de ativos ou um setor inteiro, ambos os tipos de \emph{valuation} podem diferir entre si.

  \hypertarget{anuxe1lise-por-muxfaltiplos}{%
  \subsection{Análise por múltiplos}\label{anuxe1lise-por-muxfaltiplos}}

  Múltiplos, efetivamente, são uma tentativa de reduzir empresas a ``fatores comuns'', para que possam ser feitas comparações tão precisas quanto possíveis. No geral, valores podem ser padronizados relativo aos lucros que as firmas geram, aos valores contábeis; ou valores de substituição das firmas em si mesmas, às receitas que firmas geram, ou para medidas que são específicas para as firmas em um setor.

  Pelo interesse de se desenvolver melhores ferramentas de filtragem, o estudo de tais indicadores foi iniciado. Vale, então, comentar a respeito da precisão histórica de tais múltiplos.

  Existiram, naturalmente, estudos que relacionaram múltiplos com fundamentais de \emph{valuation} (BEAVER; MORSE, \protect\hyperlink{ref-beaver1978}{1978}; PEASNELL, \protect\hyperlink{ref-peasnell1982}{1982}). De fato, há uma forte conexão entre o valor contábil e retorno sobre patrimônio líquido, como notado por Wilcox (\protect\hyperlink{ref-wilcox1984}{1984}), fazendo o argumento de que ações baratas são aquelas que são vendidas a múltiplos de valor contábil baixos enquanto mantém retornos sobre o patrimônio líquidos altos ou, no mínimo, razoáveis. Na verdade, Penman (\protect\hyperlink{ref-penman1996}{1996}) também traça uma distinção entre múltiplos de preço sobre lucro e múltiplos de valor contábil no que tange a sua ligação com o retorno sobre o patrimônio líquido, ao mencionar que, à medida que múltiplos de PBV aumentam com o ROE, a relação entre múltiplos de P/L e ROE é mais fraca. Múltiplos de lucros por ação são os melhores em explicar diferenças em precificação, múltiplos de vendas e fluxos de caixa operacionais são os piores, e múltiplos de valor contábil e EBITDA tendem a ficar no meio (LIE; LIE, \protect\hyperlink{ref-lie2002}{2002}; LIU; NISSIM; THOMAS, \protect\hyperlink{ref-liu2002}{2002}, \protect\hyperlink{ref-liu2007}{2007}). Curiosamente, existe ainda uma relação inversa entre volatilidade do mercado e múltiplos de P/L (STOHS; MAUER, \protect\hyperlink{ref-stohs1996}{1996}), demostrando a aversão ao risco do investidor.

  \hypertarget{estudo-de-caso}{%
  \chapter{Estudo de caso}\label{estudo-de-caso}}

  Placeholder

  \hypertarget{breve-dossiuxea-da-copel}{%
  \section{Breve dossiê da COPEL}\label{breve-dossiuxea-da-copel}}

  \hypertarget{histuxf3ria}{%
  \subsection{História}\label{histuxf3ria}}

  \hypertarget{core-business}{%
  \subsection{\texorpdfstring{\emph{Core business}}{Core business}}\label{core-business}}

  \hypertarget{gerauxe7uxe3o}{%
  \subsubsection{Geração}\label{gerauxe7uxe3o}}

  \hypertarget{transmissuxe3o}{%
  \subsubsection{Transmissão}\label{transmissuxe3o}}

  \hypertarget{distribuiuxe7uxe3o}{%
  \subsubsection{Distribuição}\label{distribuiuxe7uxe3o}}

  \hypertarget{outros}{%
  \subsubsection{Outros}\label{outros}}

  \hypertarget{cuxe1lculo-do-valuation-intruxednseco}{%
  \section{\texorpdfstring{Cálculo do \emph{valuation} intrínseco}{Cálculo do valuation intrínseco}}\label{cuxe1lculo-do-valuation-intruxednseco}}

  \hypertarget{o-custo-de-capital-muxe9dio-ponderado-wacc}{%
  \subsection{O custo de capital médio ponderado (WACC)}\label{o-custo-de-capital-muxe9dio-ponderado-wacc}}

  \hypertarget{custo-de-capital-pruxf3prio}{%
  \subsubsection{Custo de capital próprio}\label{custo-de-capital-pruxf3prio}}

  \hypertarget{custo-de-capital-de-terceiros}{%
  \subsubsection{Custo de capital de terceiros}\label{custo-de-capital-de-terceiros}}

  \hypertarget{fluxo-de-caixa-descontado}{%
  \subsubsection{Fluxo de caixa descontado}\label{fluxo-de-caixa-descontado}}

  \hypertarget{cuxe1lculo-do-valuation-relativo}{%
  \section{\texorpdfstring{Cálculo do \emph{valuation} relativo}{Cálculo do valuation relativo}}\label{cuxe1lculo-do-valuation-relativo}}

  \hypertarget{margem-bruta}{%
  \subsection{Margem bruta}\label{margem-bruta}}

  \hypertarget{lucros-antes-de-juros-e-impostos-ebit}{%
  \subsection{Lucros antes de juros e impostos (EBIT)}\label{lucros-antes-de-juros-e-impostos-ebit}}

  \hypertarget{margem-luxedquida}{%
  \subsection{Margem líquida}\label{margem-luxedquida}}

  \hypertarget{razuxe3o-preuxe7olucro-pe}{%
  \subsection{Razão preço/lucro (P/E)}\label{razuxe3o-preuxe7olucro-pe}}

  \hypertarget{retorno-sobre-patrimuxf4nio-luxedquido-roe}{%
  \subsection{Retorno sobre patrimônio líquido (ROE)}\label{retorno-sobre-patrimuxf4nio-luxedquido-roe}}

  \hypertarget{comparauxe7uxe3o-com-empresas-do-setor}{%
  \subsection{Comparação com empresas do setor}\label{comparauxe7uxe3o-com-empresas-do-setor}}

  \hypertarget{conclusuxe3o}{%
  \chapter{Conclusão}\label{conclusuxe3o}}

  \hypertarget{referuxeancias-bibliogruxe1ficas}{%
  \chapter*{Referências Bibliográficas}\label{referuxeancias-bibliogruxe1ficas}}
  \addcontentsline{toc}{chapter}{Referências Bibliográficas}

  Placeholder

  \hypertarget{refs}{}
  \begin{cslreferences}
  \leavevmode\hypertarget{ref-arnold2008}{}%
  ARNOLD, G. \textbf{Corporate financial management}. New York City, NY: Pearson Education, 2008.

  \leavevmode\hypertarget{ref-beaver1978}{}%
  BEAVER, W.; MORSE, D. What determines price-earnings ratios? \textbf{Financial Analysts Journal}, v. 34, n. 4, p. 65--76, 1978.

  \leavevmode\hypertarget{ref-damodaran2007}{}%
  DAMODARAN, A. \textbf{Valuation approaches and metrics: a survey of the theory and evidence}. {[}s.l.{]} Now Publishers Inc, 2007.

  \leavevmode\hypertarget{ref-damodaran2012}{}%
  DAMODARAN, A. \textbf{Investment philosophies: successful strategies and the investors who made them work}. Hoboken, NJ: John Wiley \& Sons, 2012.

  \leavevmode\hypertarget{ref-fama2004}{}%
  FAMA, E. F.; FRENCH, K. R. The capital asset pricing model: Theory and evidence. \textbf{Journal of economic perspectives}, v. 18, n. 3, p. 25--46, 2004.

  \leavevmode\hypertarget{ref-fisher1930}{}%
  FISHER, I. \textbf{Theory of interest: as determined by impatience to spend income and opportunity to invest it}. New York City, NY: Augustusm Kelly Publishers, Clifton, 1930.

  \leavevmode\hypertarget{ref-gordon1959}{}%
  GORDON, M. J. Dividends, earnings, and stock prices. \textbf{The review of economics and statistics}, p. 99--105, 1959.

  \leavevmode\hypertarget{ref-gordon1956}{}%
  GORDON, M. J.; SHAPIRO, E. Capital equipment analysis: the required rate of profit. \textbf{Management science}, v. 3, n. 1, p. 102--110, 1956.

  \leavevmode\hypertarget{ref-graham2016}{}%
  GRAHAM, B. \textbf{O investidor inteligente}. Rio de Janeiro: HarperCollins Brasil, 2016.

  \leavevmode\hypertarget{ref-lie2002}{}%
  LIE, E.; LIE, H. J. Multiples used to estimate corporate value. \textbf{Financial Analysts Journal}, v. 58, n. 2, p. 44--54, 2002.

  \leavevmode\hypertarget{ref-liu2002}{}%
  LIU, J.; NISSIM, D.; THOMAS, J. Equity valuation using multiples. \textbf{Journal of Accounting Research}, v. 40, n. 1, p. 135--172, 2002.

  \leavevmode\hypertarget{ref-liu2007}{}%
  LIU, J.; NISSIM, D.; THOMAS, J. Is cash flow king in valuations? \textbf{Financial Analysts Journal}, v. 63, n. 2, p. 56--68, 2007.

  \leavevmode\hypertarget{ref-markowitz1952}{}%
  MARKOWITZ, H. Portfolio selection. \textbf{The Journal of Finance}, v. 7, n. 1, p. 77--91, 1952.

  \leavevmode\hypertarget{ref-modigliani1958}{}%
  MODIGLIANI, F.; MILLER, M. H. The cost of capital, corporation finance and the theory of investment. \textbf{The American economic review}, v. 48, n. 3, p. 261--297, 1958.

  \leavevmode\hypertarget{ref-peasnell1982}{}%
  PEASNELL, K. V. Some formal connections between economic values and yields and accounting numbers. \textbf{Journal of Business Finance \& Accounting}, v. 9, n. 3, p. 361--381, 1982.

  \leavevmode\hypertarget{ref-penman1996}{}%
  PENMAN, S. H. The articulation of price-earnings ratios and market-to-book ratios and the evaluation of growth. \textbf{Journal of accounting research}, v. 34, n. 2, p. 235--259, 1996.

  \leavevmode\hypertarget{ref-sharpe1964}{}%
  SHARPE, W. F. Capital asset prices: A theory of market equilibrium under conditions of risk. \textbf{The journal of finance}, v. 19, n. 3, p. 425--442, 1964.

  \leavevmode\hypertarget{ref-stohs1996}{}%
  STOHS, M. H.; MAUER, D. C. The determinants of corporate debt maturity structure. \textbf{Journal of business}, p. 279--312, 1996.

  \leavevmode\hypertarget{ref-wilcox1984}{}%
  WILCOX, J. W. The P/B-roe valuation model. \textbf{Financial Analysts Journal}, v. 40, n. 1, p. 58--66, 1984.
  \end{cslreferences}
  \backmatter
  \bibliographystyle{coppe-unsrt}
  \bibliography{thesis}

  %\appendix
  %\include{appenA}
\end{document}

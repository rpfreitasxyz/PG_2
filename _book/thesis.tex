\documentclass[grad,numbers]{coppe}
\usepackage[utf8]{inputenc}
\usepackage{amsmath,amssymb}
\usepackage{hyperref}

\makelosymbols
\makeloabbreviations



  \RequirePackage[english, brazil]{babel}


\newlength{\cslhangindent}
\setlength{\cslhangindent}{1.5em}
\newenvironment{cslreferences}%
  {}%
  {\par}

\providecommand{\tightlist}{%
  \setlength{\itemsep}{0pt}\setlength{\parskip}{0pt}}
\usepackage{longtable}
\usepackage{booktabs}
\begin{document}
  \title{\emph{Valuation} Intrínseco e Relativo: O estudo de caso da COPEL}
  \foreigntitle{Intrinsic and Relative Valuation: The case study of COPEL}
    \author{Rafael Pinto}{de Freitas}
  
    \advisor{Prof.}{José Roberto}{Ribas}{D.Sc.}
    \advisor{Prof.}{Nome do Segundo Orientador}{Sobrenome}{Ph.D}
  

    \examiner{Prof.}{José Roberto Ribas}{D.Sc.}
    \examiner{Prof.}{Nome Completo do Segundo Examinador}{Ph.D}
    \examiner{Prof.}{Nome Completo do Terceiro Examinador}{Ph.D}
  
  \department{EPR}
  \date{11}{2020}

    \keyword{Valuation}
    \keyword{Análise de investimentos}
  
  \maketitle

  \frontmatter
  \dedication{\begin{quote}
Judge a man by his questions rather than by his answers.

\hfill --- Voltaire
\end{quote}}
    \chapter*{Agradecimentos}
  Agradeço pela oportunidade de cursar um ensino superior de qualidade de forma pública. Mesmo com suas diversas limitações e imperfeições, a República brasileira segue em frente com a mensagem de democratização do conhecimento. É somente por meio desta que podemos nos defender contra a tirania vil da ignorância. Dessa forma, estou em dívida com a sociedade; com todos que permitiram minha entrada e estadia no curso de Engenharia de Produção pela UFRJ. Uma dívida monumental, se pensada pela ótica dos benefícios. Espero retornar o investimento em breve, a começar de forma humilde com este trabalho de conclusão de curso. Boa leitura!

  Lorem ipsum
  \begin{abstract}
Sit urna lacus aenean euismod morbi integer mauris ligula euismod. Massa leo nunc rutrum non vulputate viverra erat aliquet torquent. Dictumst inceptos litora diam dui eu non sodales eget metus? Mollis faucibus justo class class nulla vestibulum consequat purus.

Sit est ligula massa massa. Lectus parturient vehicula luctus nisl facilisis iaculis sagittis euismod ornare ut platea! Vestibulum et cras nostra luctus morbi cubilia et ante ornare luctus commodo facilisis nam. Lobortis ligula dictum tortor facilisis ante gravida habitasse cras laoreet. Vehicula pharetra vulputate non magna ut interdum habitant quam et class elementum arcu!

Adipiscing nulla laoreet magna dignissim nostra phasellus lacinia elementum est id! Rutrum arcu aliquet torquent porttitor ligula eget dictumst aenean. Lacus dictumst phasellus sed lobortis leo convallis velit mi imperdiet. Ultricies convallis id vestibulum morbi rutrum tortor diam volutpat euismod montes enim cras eros luctus duis rutrum integer.

Consectetur platea augue vitae vitae integer ad tincidunt torquent ac. Pharetra malesuada odio non lobortis dis aliquet arcu nascetur magna porttitor. Lacinia curabitur primis ligula magna sociosqu hendrerit sociosqu risus cubilia. Arcu potenti mi pellentesque nulla per varius vitae lectus pellentesque! Tempor.
  \end{abstract}
  \begin{foreignabstract}
Sit urna lacus aenean euismod morbi integer mauris ligula euismod. Massa leo nunc rutrum non vulputate viverra erat aliquet torquent. Dictumst inceptos litora diam dui eu non sodales eget metus? Mollis faucibus justo class class nulla vestibulum consequat purus.

Sit est ligula massa massa. Lectus parturient vehicula luctus nisl facilisis iaculis sagittis euismod ornare ut platea! Vestibulum et cras nostra luctus morbi cubilia et ante ornare luctus commodo facilisis nam. Lobortis ligula dictum tortor facilisis ante gravida habitasse cras laoreet. Vehicula pharetra vulputate non magna ut interdum habitant quam et class elementum arcu!

Adipiscing nulla laoreet magna dignissim nostra phasellus lacinia elementum est id! Rutrum arcu aliquet torquent porttitor ligula eget dictumst aenean. Lacus dictumst phasellus sed lobortis leo convallis velit mi imperdiet. Ultricies convallis id vestibulum morbi rutrum tortor diam volutpat euismod montes enim cras eros luctus duis rutrum integer.

Consectetur platea augue vitae vitae integer ad tincidunt torquent ac. Pharetra malesuada odio non lobortis dis aliquet arcu nascetur magna porttitor. Lacinia curabitur primis ligula magna sociosqu hendrerit sociosqu risus cubilia. Arcu potenti mi pellentesque nulla per varius vitae lectus pellentesque! Tempor.
  \end{foreignabstract}
  \tableofcontents

  \listoffigures

  \listoftables

  \printlosymbols
  \printloabbreviations

  \mainmatter

  \hypertarget{introduuxe7uxe3o}{%
  \chapter{Introdução}\label{introduuxe7uxe3o}}

  Cabe, antes de começar o desenvolvimento do trabalho propriamente dito, contextualizar e justificar o trabalho, assim como explicitar aos leitores os objetivos, as limitações e a estrutura do mesmo.

  \hypertarget{contextualizauxe7uxe3o}{%
  \section{Contextualização}\label{contextualizauxe7uxe3o}}

  É necessário definir Bolsas de Valores, assim como retratar uma breve história da brasileira a partir de 1967 -- ano a partir do qual começou a se chamar Bolsa de Valores de São Paulo, a Bovespa. Segundo Assaf Neto (\protect\hyperlink{ref-assafneto2018}{2018}), as Bolsas são entidades, cujo objetivo básico é o de manter um local em condições adequadas para a realização de operações de compra e venda de títulos e valores mobiliários.

  Um ano após 1967, foi criado o principal índice de ações brasileiro: o Ibovespa. Resumidamente, este índice é uma média ponderada das ações com maior volume de negociação. Após certo tempo, foi criada a Cetip -- a Central de Custódia e de Liquidação Financeira de Títulos -- em 1984, começando a operar em 1986.

  A partir de 2007, as bolsas de valores deixaram de ser entidades sem fins lucrativos e tornaram-se empresas de capital aberto. No ano seguinte, a BM\&F e a Bovespa se uniram, resultando na criação da Bolsa de Valores, Mercadorias e Futuros de São Paulo -- a BM\&F Bovespa. Em 2017, são fundidas a BM\&F Bovespa e Cetip, dando origem à B3 S.A., sob a supervisão da Comissão de Valores Mobiliários -- esta é a bolsa brasileira atualmente.

  Em tempos contemporâneos, há um crescimento de CPFs registrados na Bolsa de Valores ano a ano, sendo em 2020 o recorde, um aumento de 47.7\% relativo a maio de 2019\footnote{Pode-se conferir a evolução dos mesmos baixando a planilha presente neste link: \url{http://www.b3.com.br/pt_br/market-data-e-indices/servicos-de-dados/market-data/consultas/mercado-a-vista/historico-pessoas-fisicas/}. Último acesso em 07 jul 2020.}.

  Com diversos setores e subsetores de atuação, são diversas as empresas de capital aberto à disposição para escolha do crescente número de investidores brasileiros -- o que não quer dizer que o investidor deve investir em todos. De fato, o investidor precisa minimizar seu risco, preferencialmente investindo em empresas que dentre outras características possuem alto \emph{payout} de dividendos, baixo grau de alavancagem; e menor variabilidade de lucros (BEAVER; KETTLER; SCHOLES, \protect\hyperlink{ref-beaver1970}{1970}).

  Empresas do setor de utilidades possuem tais características, em sua maioria; dessa forma, escolheu-se um exemplar do mesmo -- a Companhia Paranaense de Energia, COPEL -- como forma de fazer o valuation da empresa, fazer um breve resumo do setor elétrico, comparar os métodos de avaliação; e, evidentemente, exemplificar o processo ao leitor.

  \hypertarget{justificativa}{%
  \section{Justificativa}\label{justificativa}}

  Por mais que a caderneta de poupança seja o investimento preferido dos brasileiros, segundo a Associação Brasileira das Entidades dos Mercados Financeiro e de Capitais (\protect\hyperlink{ref-anbima2019}{2019}) -- esta modalidade conta com 88\% da população -- é evidente que o rendimento nominal\footnote{Nominal, pois os retornos reais são contextuais devido ao pagamento de impostos, dividendos; e, a depender do ativo, dos custos do mesmo.} do IBOV é superior no longo prazo. Pelo IPEA, pode-se obter tanto os rendimentos nominais (em \% a.m.) da poupança quanto os retornos do IBOV (também em \% a.m.), ambos em intervalos mensais desde 1990.\footnote{Ambas as séries podem ser obtidas em \url{http://www.ipeadata.gov.br/Default.aspx}. Último acesso em 10 jul 2020.} A partir daí, acumulamos as taxas recursivamente. Definindo \(i\) como a taxa de retorno nominal total, e \(i_t\) como sendo a taxa de retorno nominal no mês \(t\), temos \(i = \prod_{t=1}^n (1+i_t),\) sendo \(n\) o mês que se deseja calcular o retorno nominal total. Exposto isso, foi feito um gráfico dos retornos totais de ambas as modalidades:
  \begin{figure}[H]
  \includegraphics[width=1\linewidth]{img/ibov-poupanca} \caption{Uma comparação entre o IBOV e a poupança.}\label{fig:unnamed-chunk-1}
  \end{figure}
  Com o passar do tempo, entretanto, a maturação no mercado financeiro pode, naturalmente, levar o investidor a se interessar por retornos acima do mercado -- fundos de índice, por definição, impossibilitam o objetivo. Isso leva a uma exploração de diferentes classes de ativos, desde ações ordinárias a fundos de investimento em ações. Um investidor, entretanto, há de ter em mente que raramente fundos de investimento com administração ativa, no longo prazo, superam os retornos dos fundos de índice, em termos reais (BOGLE, \protect\hyperlink{ref-bogle2015}{2015}).

  Assim sendo, caso o investidor deseja ter sucesso, é importante que as escolhas de ativos sejam racionais. Ao menos, tão racionais quanto possível forem para humanos. De fato, indivíduos estão sujeitos a uma racionalidade restrita (SIMON, \protect\hyperlink{ref-simon1997}{1997}), o que leva a diversas heurísticas -- inclusive, mas não limitadas a: enfatizar evidências que apoiam visões próprias (KLAYMAN, \protect\hyperlink{ref-klayman1995}{1995}), superestimar probabilidades por maior ``disponibilidade'' em memória (SCHWARZ et al., \protect\hyperlink{ref-schwarz1991}{1991}); e superestimar a própria habilidade, quando se é novato, assim como subestimar, quando se é um \emph{expert} (KRUGER; DUNNING, \protect\hyperlink{ref-kruger1999}{1999}) -- para simplificar o processo de raciocínio e, permitir, assim, que o agente consiga satisfazer as restrições -- tempo, recursos, dentre outros -- para a tomada de decisões.

  De fato, a abordagem de realizar tanto o \emph{valuation} intrínseco quanto o relativo de uma empresa é, além de uma interessante comparação entre métodos, uma forma de evitar a ``síndrome do homem com martelo'', popularizado por Munger (\protect\hyperlink{ref-munger2006}{2006}). Este cita um provérbio, que diz: ``Para um homem com um martelo, todo problema se parece com um prego.''

  \hypertarget{objetivos}{%
  \section{Objetivos}\label{objetivos}}

  São os principais objetivos do trabalho (1) fazer o \emph{valuation} da Companhia Paranaense de Energia através de, no mínimo, dois métodos de valuation, sendo no mínimo um deles intrísenco e no mínimo, um relativo e (2) realizar a comparação entre os resultados dos métodos.

  Com o desenvolver do trabalho, poderão ser percebidas outras motivações, entretanto seriam estas consideradas secundárias.

  \hypertarget{limitauxe7uxf5es}{%
  \section{Limitações}\label{limitauxe7uxf5es}}

  Este trabalho se limita a prover um breve prospecto do cenário energético brasileiro, assim como possíveis desenvolvimentos. Não serão discutidas políticas energéticas e afins. Se limita, também, a tomar como verdadeira a teoria moderna do portfólio como exposta por Markowitz (\protect\hyperlink{ref-markowitz1952}{1952}), para o cálculo do custo de capital. Não será discutido economia comportamental, nem modelos mais sofisticados para tal cálculo.

  \hypertarget{estrutura-do-trabalho}{%
  \section{Estrutura do trabalho}\label{estrutura-do-trabalho}}

  O trabalho possui, em sua integridade, cinco capítulos.

  O primeiro capítulo é uma introdução ao restante do trabalho, e é efetivamente um resumo do que o leitor verá pela frente.

  O segundo capítulo é uma examinação do setor energético brasileiro, a ser feito pela leitura e examinação do Plano Decenal de Expansão de Energia (PDE) e o Plano Nacioanl de Energia (PNE), ambos elaborados pela Empresa de Pesquisa Energética (EPE). Através destes, podemos ter uma melhor noção do setor no qual a empresa está inserido, possibilitando um melhor cálculo e previsão dos fluxos de caixa.

  O terceiro capítulo é a documentação do referencial teórico utilizado, a ser escrito seguindo uma lógica linear, de forma tal que possa também ser visto como uma metodologia, com exemplos para auxiliar o leitor.

  O quarto capítulo é o estudo de caso de fato. Será dada uma contextualização da empresa, assim como a aplicação dos métodos discutidos.

  O quinto e último capítulo é a conclusão, em que será feita a exposição dos resultados, assim como a comparação entre os métodos de \emph{valuation} discutidos durante o texto.

  \hypertarget{o-mercado-de-energia}{%
  \chapter{O mercado de energia}\label{o-mercado-de-energia}}

  É importante, antes de partir para o estudo de caso em questão, estabelecer o contexto do trabalho em questão. De fato, neste capítulo serão tratados os fundamentais do mercado de energia no Brasil, desde os órgãos aos estudos principais do setor.

  \hypertarget{uxf3rguxe3os-presentes-no-estudo}{%
  \section{Órgãos presentes no estudo}\label{uxf3rguxe3os-presentes-no-estudo}}

  Nesta seção, serão tratados os cinco principais órgãos do setor, sendo eles: (1) o Ministério de Minas e Energia; (2) a Agência Nacional de Energia Elétrica; (3) o Operador Nacional do Sistema Elétrico; (4) a Câmara Comercializadora de Energia Elétrica; e (5) a Empresa de Pesquisa Energética.

  \hypertarget{mme}{%
  \subsection{MME}\label{mme}}

  O Ministério de Minas e Energia (MME) foi criado em 1960, no governo do presidente Juscelino Kubitschek. Assim seguiu por 30 anos, até ser extinto em 1990 e recriado em 1992.

  Em 1997, foi criado o Conselho Nacional de Política Energética (CNPE), vinculado à Presidência da República e presidido pelo ministro de Minas e Energia, com a atribuição de propor ao presidente da República políticas nacionais e medidas para o setor.

  Logo em seguida, em 2003, foram definidas as competências do MME como sendo as áreas de (1) geologia, recursos minerais e energéticos; (2) aproveitamento da energia hidráulica; mineração e metalurgia; e (3) petróleo, combustível e energia elétrica.

  No ano seguinte foi criado o Comitê de Monitoramento do Setor Elétrico (CMSE), cuja função é acompanhar e avaliar permanentemente a continuidade e a segurança do suprimento eletroenergético em todo o território nacional. No mesmo ano, foi permitida também a criação da Empresa de Pesquisa Energética (EPE), vinculada ao Ministério, que tem por finalidade prestar serviços na área de estudos e pesquisas destinadas a subsidiar o planejamento do setor energético.

  O MME tem como empresas vinculadas a Eletrobras e Petrobras. Estão, também, vinculadas algumas autarquias ao Ministério, dentre elas: a Agência Nacional de Energia Elétrica (ANEEL); Agência Nacional do Petróleo, Gás Natural e Biocombustíveis (ANP); e a Agência Nacional de Mineração (ANM).

  \hypertarget{aneel}{%
  \subsection{ANEEL}\label{aneel}}

  A Agência Nacional de Energia Elétrica (ANEEL) é uma autarquia vinculada ao Ministério de Minas e Energia. Tem como finalidade regular e fiscalizar a produção, transmissão, distribuição e comercialização de energia elétrica, de acordo com a legislação e em conformidade com as diretrizes e as políticas do governo federal. A autarquia foi criada em dezembro de 1996, durante o mandato de Fernando Henrique Cardoso.

  Cabe à ANEEL, dentre outras competências: (1) implementar as políticas e diretrizes do governo federal para a exploração da energia elétrica e o aproveitamento dos potenciais hidráulicos; (2) estabelecer as tarifas para o suprimento de energia elétrica realizado às concessionárias e permissionárias de distribuição; e (3) fazer a gestão dos contratos de concessão ou de permisão de serviços públicos de energia elétrica e fiscalizar, diretamente ou mediante convênios com órgãos estaduais, as concessões, as permissões e a prestação dos serviços de energia elétrica.

  Pode-se conferir a lista completa de atribuições pelo art. 3° da lei n°
  9.427/96.

  \hypertarget{ons}{%
  \subsection{ONS}\label{ons}}

  O Operador Nacional do Sistema Elétrico (ONS) é uma entidade privada sem fins lucrativos que é responsável pela coordenação e controle da operação de instalações de geração e tranmissão de energia elétrica do Sistema Interligado Nacional (SIN), sob fiscalização da ANEEL. O órgão foi criado em agosto de 1998, sob mandato do presidente Fernando Henrique Cardoso.

  \hypertarget{ccee}{%
  \subsection{CCEE}\label{ccee}}

  A Câmara de Comercialização de Energia Elétrica (CCEE) tem por finalidade viabilizar a comercialização de energia elétrica no mercado de energia brasileiro. De fato, esta efetua a contabilização e a liquidação financeira das operações realizadas no mercado de curto prazo. As regras e procedimentos que regulam as atividades realizadas da CCEE são aprovados pela ANEEL.

  Foi criada em agosto de 2004, sob mandato do presidente Luiz Inácio Lula da Silva, sucedendo ao antigo Mercado Atacadista de Energia.

  \hypertarget{epe}{%
  \subsection{EPE}\label{epe}}

  A Empresa de Pesquisa Energética (EPE) é uma empresa pública vinculada ao Ministério de Minas e Energia, criada em março de 2004, sob mandato do presidente Luiz Inácio Lula da Silva. Tem por finalidade prestar serviços na área de estudos e pesquisas destinadas a subsidiar o planejamento do setor energético.

  Cabe à EPE, dentre outras atribuições: (1) apresentar ao CNPE, anualmente, os Planos Decenais de Expansão (PDE), assim como a cada dois anos os Planos Nacionais de Energia (PNE); (2) identificar e quantificar os potenciais de recursos energéticos; e (3) elaborar e publicar o balanço energético nacional.

  A lista completa de atribuições pode ser vista no art. 4° da lei n° 10.847/04.

  Uma das principais motivações para a criação da EPE foram os racionamentos e apagões ocorridos no início da década, ocorridos durante o governo de Fernando Henrique Cardoso, atribuídos em parte á carência de planejamento. Embora seja uma entidade independente, é vinculada ao Ministério de Minas e Energia.

  \hypertarget{o-fluxo-de-energia}{%
  \section{O fluxo de energia}\label{o-fluxo-de-energia}}

  Atualmente, o setor elétrico brasileiro tem uma estrutura predominantemente unidirecional nos fluxos de energia. Assim, são tipicamente dividos em geração, transmissão, comercialização e distribuição. Por se tratarem do final da cadeia de valor convencional, os consumidores então têm efetivamente um comportamento passivo.

  A geração de energia elétrica é a transformação da energia primária -- petróleo, carvão mineral, dentre outros -- em energia elétrica. No Brasil, predomina-se a geração hidráulica devido ao grande potencial hidroenergético dos nossos rios.

  A transmissão efetua o transporte da energia gerada até os centros consumidores de carga. Nessa parte, o sistema brasileiro possui uma particularidade, que é o fato de que os grandes centros de consumidores ficam localizados longe da grande geração energética. Isso faz com que o Brasil tenha uma grande quantidade de linhas de transmissão com algumas centenas de quilômetros.

  A comercialização de energia é atualmente realizada em dois ambientes diferentes: (1) no Ambiente de Contratação Livre, que é destinado ao atendimento de consumidores livres -- o consumidor que pode optar pela compra de energia elétrica junto a qualquer fornecedor, que é atendido em qualquer tensão e com demanda contratada mínima de 3 MW, segundo a Resolução da ANEEL n° 264 e 456; e (2) no Ambiente de Contratação Regulada, que é destinado ao atendimento de consumidores cativos por meio das distribuidoras, sendo estas supridas por geradores estatais ou independentes que vendem energia em leilões públicos.

  A distribuição é o setor responsável por receber a energia da transmissão e distribuí-la para os centros consumidores.

  Salvo no caso de \emph{holdings}, empresas só podem atuar em uma das áreas ex-comercialização.

  \hypertarget{estudos-e-projeuxe7uxf5es-de-longo-prazo}{%
  \section{Estudos e projeções de longo prazo}\label{estudos-e-projeuxe7uxf5es-de-longo-prazo}}

  A EPE cumpre sua finalidade de braço técnico de estudos do setor energético produzindo relatórios e planos de expansão a respeito do setor elétrico. Por se tratar de um \emph{valuation} de uma empresa do setor, é importante ponderar a respeito das ações tomadas e o parecer de \emph{experts} a respeito do futuro do setor como um todo, a fim de aprimorar nossas estimativas da taxa de crescimento.

  São os dois principais estudos, nesse sentido, (1) o Plano Nacional de Energia; (2) e o Plano Decenal de Expansão de Energia

  \hypertarget{plano-nacional-de-energia-pne}{%
  \subsection{Plano Nacional de Energia (PNE)}\label{plano-nacional-de-energia-pne}}

  O PNE é um conjunto de estudos que dão suporte ao desenho da estratégia de longo prazo do governo em relação à expansão do setor de energia. A estratégia, por sua vez, consiste em um conjunto de recomendações e diretrizes a serem seguidas na definição das ações e iniciativas a serem implementadas ao longo do horizonte de tempo prescrito. A revisão do PNE deve ser conduzida sempre que houver necessidade de alteração na estratégia de longo prazo do tomador de decisão relevante, sejam por razões ordinárias ou extraordinárias.

  Os PNEs têm relação com os PDEs. Embora haja similaridades, o PNE é um instrumento de planejamento com atribuições distintas do PDE. Em primeiro lugar, o PNE é um documento com visão mais estratégica. Neste contexto, seu enfoque consiste em embasar o posicionamento do governo de modo a orientar e direcionar as estratégias dos agentes do setor de forma a se atingir os objetivos gerais da expansão no longo prazo, com adequação dos recursos, com confiabilidade, modicidade e sustentabilidade. Além disso, o PNE é o alicerce a partir de qual todos os Planos, Políticas, Programas e Iniciativas são elaborados. O PNE é, portanto, um farol que orienta para onde os PDEs devem indicar a expansão do setor de energia no horizonte decenal. Por fim, enquanto o PDE trabalha com um cenário de referência e análises de sensibilidade, anualmente revisto, o PNE deve tratar com mais cenários.

  No caso, foi feito o estudo do PNE 2050 (EPE, \protect\hyperlink{ref-epe2020}{2020}), e assim será entregue um pequeno sumário executivo do mesmo a seguir.

  Foram elaborados dois grandes cenários, formando um cone de incertezas para o desenho da estratégia de longo prazo: o primeiro, chamado Desafio da Expansão, reflete requisitos de expansão do setor de energia para atendimento a um crescimento da demanda de energia mais expressivo. No segundo, chamado de Estagnação, analisam-se as consequências de um cenário em que o consumo de energia per capita mantém-se inalterado ao longo de todo o período. O foco do relatório está voltado para o cenário Desafio da Expansão.

  Os estudos do PNE apontam para um potencial energético de quase 280 bilhões de tep (toneladas equivalentes de petróleo) no horizonte até 2050. A demanda cresce de 300 milhões de tep para 600 milhões de tep e, ao longo de trinta e cinco anos, essa trajetória representa uma demanda de energia total acumulada do período equivalente a pouco menos de 15 bilhões de tep. Tamanha discrepância entre potencial de recursos e a demanda de energia estimada gera uma situação distinta daquela vivida ao longo especialmente da metade do século XX, quando o País viveu grandes crises de energia. Dessa forma, imagina-se uma administração da abundância.

  O cenário Desafio da Expansão considera algumas premissas, que serão levadas em conta para o uso no estudo de caso. São essas: (1) crescimento médio de PIB de 3.1\% a.a. e 2.8\% de PIB per capita; (2) a população brasileira manterá a tendência de crescer a taxas cada vez menores; (3) o consumo potencial de energia elétrica do País pode atingir até 3 vezes o patamar do ano base; (4) a demanda de energia elétrica a ser atendida por geração centralizada cresce até 2.5 vezes em relação aos valores do ano base, mesmo com crescimento de geração distribuída (GD), autoprodução, energia solar térmica e eficiência energética no período; (5) e o consumo de energia e de eletricidade per capita aumenta, a despeito da contribuição da eficiência energética no horizonte até 2050.

  \hypertarget{plano-decenal-de-expansuxe3o-de-energia-pde}{%
  \subsection{Plano Decenal de Expansão de Energia (PDE)}\label{plano-decenal-de-expansuxe3o-de-energia-pde}}

  O PDE tem o intuito de de indicar as perspectivas, sob a ótica do Governo, da expansão do setor de energia no horizonte de dez anos, dentro de uma visão integrada para os diversos energéticos. Tal visão permite, então, extrair importantes elementos para o planejamento do setor, com benefícios em termos de aumento de confiabilidade, redução de custos de produção e mitigação de impactos ambientais.

  Como o PDE tem intuitos mais operacionais do que estratégicos, serão utilizadas as premissas e estudos do PNE para nortear as decisões do presente \emph{valuation}.

  \hypertarget{referencial-teuxf3rico}{%
  \chapter{Referencial teórico}\label{referencial-teuxf3rico}}

  Nesta seção será feita uma consideração a respeito dos métodos e conceitos utilizados ao longo do estudo. É de interesse do leitor prestar especial atenção ao enunciado abaixo, uma vez que é um breve alicerce teórico que serve não apenas para esse estudo, como para diversos outros similares. Recomenda-se, ainda, a leitura de Damodaran (\protect\hyperlink{ref-damodaran2007}{2007}) para se ter um panorama de diversos outros modelos a serem aplicados, assim como uma discussão a respeito de seus usos e efetividade.

  \hypertarget{valuation-intruxednseco}{%
  \section{\texorpdfstring{\emph{Valuation} intrínseco}{Valuation intrínseco}}\label{valuation-intruxednseco}}

  Comecemos discutindo brevemente a respeito do que é possuir ``valor intrínseco''. Este é um conceito filosófico, em que o valor de um objeto ou projeto é derivado de, e por si só -- em outras palavras, livre de fatores externos. Analistas financeiros constroem modelos para estimar o que se imagina ser o valor intrínseco de uma empresa sem considerar o seu valor de mercado em determinado dia.

  Naturalmente, o mercado, no curto prazo, está sujeito a flutuações que podem ser atribuídas a diversos fatores, desde manipulação de preços em papéis mais ilíquidos a pensamento de manada por parte de investidores. Cabe, nesse momento, utilizar a analogia popularizada por Graham (\protect\hyperlink{ref-graham2016}{2016}), do Sr.~Mercado:
  \begin{quote}
  ``Imagine que você possui uma participação pequena em uma companhia de capital fechado que lhe custou US\$1.000. Um de seus sócios, chamado Sr.~Mercado, é de fato muito prestativo. Todo dia ele lhe informa o que pensa ser o valor de sua participação e, além disso, se dispõe a comprar de você ou vender a você uma participação adicional naquelas bases. Às vezes, sua ideia de valor parece plausível e justificada pela evolução e pelas perspectivas do negócio da forma como você as conhece. Por outro lado, o Sr.~Mercado deixa frequentemente o entusiasmo ou o receio tomar conta dele e o valor proposto por ele lhe parece pura bobagem.
  Se você é um investidor prudente ou um empresário inteligente, deixaria as comunicações diárias do Sr.~Mercado influenciarem sua opinião sobre o valor de uma participação de US\$1.000 na companhia? Só se você concordasse com ele ou então desejasse negociar com ele. Você pode ficar feliz em vender para ele quando ele cota um preço ridiculamente alto e igualmente feliz em comprar dele quando seu preço é baixo. No entanto, no resto do tempo, você seria mais esperto se formulasse suas próprias ideias acerca do valor de sua carteira com base nos relatório completos da companhia sobre suas operações e posições financeiras.''
  \end{quote}
  Dessa forma, a discrepância entre preço de mercado e a estimativa do valor intrísenco feita por um analista torna-se uma medida para oportunidade de investimento. Aqueles que considerarem tais modelos como estimativas razoáveis de valor intrínseco, e que tomarem ação baseando-se nessas estimativas, são conhecidos como investidores de valor (DAMODARAN, \protect\hyperlink{ref-damodaran2012}{2012}).

  \hypertarget{anuxe1lise-de-fluxo-de-caixa-descontado}{%
  \subsection{Análise de Fluxo de Caixa Descontado}\label{anuxe1lise-de-fluxo-de-caixa-descontado}}

  A análise de fluxos de caixa descontados (DCF) é um método de se descobrir o valor de uma ação, projeto, empresa, ou ativo usando os conceitos do valor temporal do dinheiro. Para se aplicar o método, todos os fluxos de caixa futuros são estimados e descontados ao se utilizar o custo de capital para dar seus valores presentes. A soma de todos os futuros fluxos de caixa, tanto de entrada quanto de saída, resulta no valor presente líquido, que é tomado como o valor dos fluxos de caixa em questão, no momento.

  Seguindo a queda do mercado em 1929, o método ganhou popularidade para se avaliar ações. De fato, provavelmente o primeiro a formalizar a expressão do método em termos econômicos modernos foi Fisher (\protect\hyperlink{ref-fisher1930}{1930}).

  O valor presente líquido pode ser expresso matematicamente como:
  \[
  VPL = \sum_{i=0}^N\frac{FC_t}{(1+r)^t}
  \]
  onde \(FC_t\) é o fluxo de caixa no tempo \(t\) e \(r\) é a taxa de desconto. Naturalmente, para que o somatório acima esteja correto, assume-se que a taxa de desconto permaneça constante através do período todo. Caso assuma-se que o fluxo de caixa continue indefinidamente, a previsão finita é geralmente combinada com a premissa de um crescimento constante de fluxo de caixa além do período de projeção discreto -- a dita perpetuidade. Matematicamente:
  \[
  VPL = \sum_{i=0}^N\frac{FC_t}{(1+r)^t} + \frac{FC_{N+2}}{(1+r)^{N+1}(r-g)}
  \]
  onde o somatório é o período de crescimento normal, e além do somatório, temos o fluxo de caixa em perpetuidade, sendo descontado.

  A pergunta, entretanto, repousa sobre encontrar a taxa de desconto. Diversos modelos foram apresentados com tal finalidade, sendo o mais utilizado o \emph{capital asset pricing model}, doravante mencionado pela sua abreviatura, CAPM.

  Este modelo foi introduzido por Sharpe (\protect\hyperlink{ref-sharpe1964}{1964}), desenvolvendo em cima do trabalho iniciado em diversificação e teoria moderna do portfólio (MARKOWITZ, \protect\hyperlink{ref-markowitz1952}{1952}). O CAPM leva em conta a sensibilidade de um ativo ao risco não diversificável -- também conhecido como risco sistemático, ou risco de mercado -- geralmente representado por \(\beta\). De fato, a equação é como segue:
  \[
  E(R_i) = R_f + \beta_i(E(R_m)-R_f)
  \]
  onde \(E(R_i)\) é o retorno esperado do ativo, \(R_f\) é a taxa de juros livre de risco -- oriunda geralmente de títulos do governo -- e \(E(R_m)\) é o retorno esperado do mercado. \(\beta\), como comentado, é a sensibilidade do ativo em relação ao mercado, de forma tal que:
  \[
  \beta_i = \frac{Cov(R_i, R_m)}{Var(R_m)} = \rho_{i,m}\frac{\sigma_i}{\sigma_m}
  \]
  onde \(\rho_{i,m}\) denota o coeficiente de correlação entre o investimento \(i\) e o mercado \(m\), \(\sigma_i\) é o desvio padrão para o investimento \(i\), e \(\sigma_m\) é o desvio padrão para o mercado \(m\).

  Podemos entender essa equação melhor se a rearranjarmos:
  \[
  E(R_i) = R_f + \beta_i(E(R_m)-R_f) \iff E(R_i) - R_f = \beta_i(E(R_m)-R_f)
  \]
  onde o segundo lado denota uma equivalência interessante. Dessa forma, podemos dizer que o prêmio de risco para o ativo individual é igual ao prêmio pelo risco de mercado, multiplicado pela sensibilidade do ativo (\(\beta\)).

  De fato, o modelo leva em conta diversas premissas, dentre as quais todos os investidores (ARNOLD, \protect\hyperlink{ref-arnold2008}{2008}):
  \begin{enumerate}
  \def\labelenumi{\arabic{enumi}.}
  \tightlist
  \item
    Têm por objetivo maximizar utilidades econômicas (quantidades de ativos são dadas e fixas).
  \item
    São racionais e aversos a risco.
  \item
    São amplamente diversificados sobre uma grande gama de investimentos.
  \item
    São tomadores de preço, isto é, não influenciam nos preços.
  \item
    Podem emprestar e tomar quantias ilimitadas sob a taxa livre de risco de juros.
  \item
    Fazem trocas sem custo de transação ou impostos.
  \item
    Lidam com ativos que são todos altamente diversificáveis em pequenas parcelas -- são perfeitamente divisíveis e líquidos).
  \item
    Têm expectativas homogêneas.
  \item
    Têm todas as informações disponíveis ao mesmo tempo.
  \end{enumerate}
  Naturalmente, pela quantidade e teor das premissas, este é um modelo que fortemente simplifica a realidade. De fato, pela sua lógica simples e fácil aplicabilidade, ainda é muito utilizado na indústria, embora a maior parte das aplicações utilizando-se este modelo sejam consideradas inválidas (FAMA; FRENCH, \protect\hyperlink{ref-fama2004}{2004}).

  Para os propósitos deste trabalho, entretanto, será feita uma análise de fluxos de caixa descontados utilizando-se, também, o CAPM.

  \hypertarget{modelo-de-desconto-de-dividendos}{%
  \subsection{Modelo de Desconto de Dividendos}\label{modelo-de-desconto-de-dividendos}}

  O modelo de desconto de dividendos (DDM) é um método de se fazer o \emph{valuation} de uma ação baseado na teoria de que a ação vale a soma de todos os seus pagamentos de dividendos futuros, descontados de volta ao seu valor presente líquido (VPL). A equação mais utilizada amplamente é o chamado modelo de crescimento de Gordon (GGM). É nomeada assim por causa da publicação de Gordon e Shapiro (\protect\hyperlink{ref-gordon1959}{1959}), embora tenha sido originalmente desenvolvida três anos antes (GORDON; SHAPIRO, \protect\hyperlink{ref-gordon1956}{1956}). Trata-se da equação:
  \[
  P_0 = \frac{D_1}{r-g}
  \]
  onde \(P_0\) é o valor atual da ação, \(g\) é a taxa de crescimento constante em perpetuidade esperada dos dividendos, \(r\) é o custo de capital próprio da empresa; e \(D_1\) é o valor dos dividendos do próximo ano.

  Naturalmente, existem alguns pressupostos deste modelo:
  \begin{enumerate}
  \def\labelenumi{\roman{enumi}.}
  \tightlist
  \item
    Uma taxa de crescimento constante e perpétua, menor que o custo de capital.
  \item
    A ação deve pagar dividendos regularmente; do contrário, versões mais generalizadas do modelo de desconto de dividendos devem ser usados para se descobrir o valor da ação.
  \end{enumerate}
  A partir destes pontos, temos que as violações de (i) identifica uma ação de valor negativo; e (ii) provê um valor errôneo -- caso seja levado ao extremo, uma empresa que não paga dividendos efetivamente não valeria nada.

  A solução para (i) é se considerar um modelo de desconto de dividendos de dois estágios, isto é:
  \[
  P_0 = \frac{D_0 (1+g)}{r-g} \left[1 - \frac{(1+g)^N}{(1+r)^N}\right] + \frac{D_0 (1+g)^N (1+g_\infty)}{(1+r)^N (r-g_\infty)}
  \]
  onde \(D_0\) denota os dividendos deste ano, \(g\) a taxa de crescimento esperada de curto prazo, \(g_\infty\) a taxa de crescimento de longo prazo, e \(N\) o período (em número de anos), através do qual a taxa de curto prazo é aplicada.

  Uma solução comum para (ii) seria assumir que a hipótese de Modigliani-Miller de irrelevância de dividendos (MODIGLIANI; MILLER, \protect\hyperlink{ref-modigliani1958}{1958}) seja verdadeira, e então substituir os dividendos \(D\) por \(E\), os lucros por ação. Entretanto, isso requer o uso de crescimento dos lucros, ao invés dos de dividendos, que podem ser diferentes.

  A equação de Gordon pode ser entendida como o fato de que o retorno total de uma ação é igual à soma da sua receita e seus ganhos de capital. De fato, se rearranjada, teremos que:
  \[
  P_0 = \frac{D_1}{r-g} \iff \frac{D_1}{P_0} + g = r
  \]
  o que significa que o \emph{dividend yield} (\(D_1/P_0\)) mais o crescimento (\(g\)) é igual ao custo de capital próprio (\(r\)). Ora, caso consideremos a taxa de crescimento de dividendos no modelo como um \emph{proxy} para o crescimento de lucros e, por extensão, o preço da ação e os ganhos de capital. Consideraríamos, então, o custo de capital próprio como um \emph{proxy} para o retorno total requerida pelo investidor.

  \hypertarget{valuation-relativo}{%
  \section{\texorpdfstring{\emph{Valuation} relativo}{Valuation relativo}}\label{valuation-relativo}}

  Em \emph{valuation} relativo, um determinado ativo é avaliado baseado em quão precificados estão os ativos similares no mercado. Como exemplo, uma comprador de imóveis pode, antes de realizar uma compra à vista/financiamento abrupto, pode pesquisar por imóveis similares na vizinhança. Ora, uma pessoa que coleciona selos pode fazer um julgamento de quanto vai pagar em outro selo raro ao checar preços de transações desse mesmo selo em outras épocas. Dessa forma, um investidor em potencial pode estimar o preço de uma ação a comprar fazendo uma pesquisa através da precificação de ações ``similares''.

  Pela descrição acima, existem três fatores a se considerar:
  \begin{enumerate}
  \def\labelenumi{\arabic{enumi}.}
  \tightlist
  \item
    \textbf{É necessário encontrar ativos comparáveis, precificados pelo mercado.} Esta é uma tarefa que é mais fácil de se cumprir com ativos tangíveis do que com imateriais, como ações. Frequentemente, analistas consideram outras empresas do mesmo setor como comparáveis, comparando por exemplo, empresas de utilidade com outras empresas de utilidade.
  \item
    \textbf{É importante traduzir os preços de mercado a uma variável comum.} A finalidade disso é gerar preços padronizados que sejam comparáveis. Embora isso não seja necessário com ativos idênticos, é necessário quando existe heterogenia. Considere, por exemplo, o exemplo dos imóveis. Uma casa tem 200 m² e outra, 100 m². Reduziria-se um fator à metragem. Naturalmente, com empresas acontece algo similar, geralmente reduzindo-se a múltiplos de lucros, valor contábil, dentre outros.
  \item
    \textbf{É necessário ajustar-se diferenças entre ativos.} Novamente, consideremos o exemplo da casa. Ambas possuem a mesma metragem, mas uma acabou de ser construída, e outra tem mais de 200 anos de idade. Ora, havendo essa diferença de idades, \emph{ceteris paribus}, a casa mais nova deve valer mais. Com ações, pode haver algo similar. Empresas de maior crescimento, \emph{ceteris paribus}, devem valer mais do que empresas de menor crescimento, por exemplo.
  \end{enumerate}
  Cabe comentar que existe uma diferença filosófica significativa entre as abordagens intrínseca e relativa. Através de \emph{valuation} intrínseco, tentamos estimar o valor de um ativo baseado na sua capacidade de gerar fluxos de caixa no futuro. No \emph{valuation} relativo, estamos fazendo um julgamento em quanto um ativo vale ao olharmos para o que o mercado está pagando por ativos similares -- implicitamente, estamos ``confiando'' no julgamento de valor do mercado. Dessa forma, caso o mercado esteja sistematicamente superestimando ou subestimando -- \emph{bull} e \emph{bear market}, respectivamente -- um grupo de ativos ou um setor inteiro, ambos os tipos de \emph{valuation} podem diferir entre si.

  \hypertarget{anuxe1lise-por-muxfaltiplos}{%
  \subsection{Análise por múltiplos}\label{anuxe1lise-por-muxfaltiplos}}

  Múltiplos, efetivamente, são uma tentativa de reduzir empresas a ``fatores comuns'', para que possam ser feitas comparações tão precisas quanto possíveis. No geral, valores podem ser padronizados relativo aos lucros que as firmas geram, aos valores contábeis; ou valores de substituição das firmas em si mesmas, às receitas que firmas geram, ou para medidas que são específicas para as firmas em um setor.

  Vale, então, comentar a respeito da precisão histórica de tais múltiplos. Em verdade, múltiplos de lucros por ação são os melhores em explicar diferenças em precificação, múltiplos de vendas e fluxos de caixa operacionais são os piores, e múltiplos de valor contábil e EBITDA tendem a ficar no meio (LIE; LIE, \protect\hyperlink{ref-lie2002}{2002}; LIU; NISSIM; THOMAS, \protect\hyperlink{ref-liu2002}{2002}, \protect\hyperlink{ref-liu2007}{2007}).

  \hypertarget{estudo-de-caso}{%
  \chapter{Estudo de caso}\label{estudo-de-caso}}

  \hypertarget{contextualizauxe7uxe3o-da-copel}{%
  \section{Contextualização da COPEL}\label{contextualizauxe7uxe3o-da-copel}}

  \hypertarget{histuxf3ria}{%
  \subsection{História}\label{histuxf3ria}}

  \hypertarget{core-business}{%
  \subsection{\texorpdfstring{\emph{Core business}}{Core business}}\label{core-business}}

  \hypertarget{gerauxe7uxe3o}{%
  \subsubsection{Geração}\label{gerauxe7uxe3o}}

  \hypertarget{transmissuxe3o}{%
  \subsubsection{Transmissão}\label{transmissuxe3o}}

  \hypertarget{distribuiuxe7uxe3o}{%
  \subsubsection{Distribuição}\label{distribuiuxe7uxe3o}}

  \hypertarget{outros}{%
  \subsubsection{Outros}\label{outros}}

  \hypertarget{cuxe1lculo-do-valuation-intruxednseco}{%
  \section{\texorpdfstring{Cálculo do \emph{valuation} intrínseco}{Cálculo do valuation intrínseco}}\label{cuxe1lculo-do-valuation-intruxednseco}}

  \hypertarget{o-custo-de-capital-muxe9dio-ponderado-wacc}{%
  \subsection{O custo de capital médio ponderado (WACC)}\label{o-custo-de-capital-muxe9dio-ponderado-wacc}}

  \hypertarget{custo-de-capital-pruxf3prio}{%
  \subsubsection{Custo de capital próprio}\label{custo-de-capital-pruxf3prio}}

  \hypertarget{custo-de-capital-de-terceiros}{%
  \subsubsection{Custo de capital de terceiros}\label{custo-de-capital-de-terceiros}}

  \hypertarget{fluxo-de-caixa-descontado}{%
  \subsubsection{Fluxo de caixa descontado}\label{fluxo-de-caixa-descontado}}

  \hypertarget{cuxe1lculo-do-valuation-relativo}{%
  \section{\texorpdfstring{Cálculo do \emph{valuation} relativo}{Cálculo do valuation relativo}}\label{cuxe1lculo-do-valuation-relativo}}

  \hypertarget{margem-bruta}{%
  \subsection{Margem bruta}\label{margem-bruta}}

  \hypertarget{lucros-antes-de-juros-e-impostos-ebit}{%
  \subsection{Lucros antes de juros e impostos (EBIT)}\label{lucros-antes-de-juros-e-impostos-ebit}}

  \hypertarget{margem-luxedquida}{%
  \subsection{Margem líquida}\label{margem-luxedquida}}

  \hypertarget{razuxe3o-preuxe7olucro-pe}{%
  \subsection{Razão preço/lucro (P/E)}\label{razuxe3o-preuxe7olucro-pe}}

  \hypertarget{retorno-sobre-patrimuxf4nio-luxedquido-roe}{%
  \subsection{Retorno sobre patrimônio líquido (ROE)}\label{retorno-sobre-patrimuxf4nio-luxedquido-roe}}

  \hypertarget{comparauxe7uxe3o-com-empresas-do-setor}{%
  \subsection{Comparação com empresas do setor}\label{comparauxe7uxe3o-com-empresas-do-setor}}

  \hypertarget{conclusuxe3o}{%
  \chapter{Conclusão}\label{conclusuxe3o}}

  \backmatter

  \hypertarget{referuxeancias-bibliogruxe1ficas}{%
  \chapter*{Referências Bibliográficas}\label{referuxeancias-bibliogruxe1ficas}}
  \addcontentsline{toc}{chapter}{Referências Bibliográficas}

  \markboth{References}{References}

  \label{bib:begin}
  \noindent

  \setlength{\parindent}{-0.20in}
  \setlength{\leftskip}{0.20in}
  \setlength{\parskip}{8pt}

  \hypertarget{refs}{}
  \begin{cslreferences}
  \leavevmode\hypertarget{ref-anbima2019}{}%
  ANBIMA. \textbf{Raio-X do Investidor Brasileiro}, 2019.

  \leavevmode\hypertarget{ref-arnold2008}{}%
  ARNOLD, G. \textbf{Corporate financial management}. {[}s.l.{]} Pearson Education, 2008.

  \leavevmode\hypertarget{ref-beaver1970}{}%
  BEAVER, W.; KETTLER, P.; SCHOLES, M. The association between market determined and accounting determined risk measures. \textbf{The Accounting Review}, v. 45, n. 4, p. 654--682, 1970.

  \leavevmode\hypertarget{ref-bogle2015}{}%
  BOGLE, J. C. \textbf{Bogle on mutual funds: New perspectives for the intelligent investor}. New Jersey: John Wiley \& Sons, 2015.

  \leavevmode\hypertarget{ref-damodaran2007}{}%
  DAMODARAN, A. \textbf{Valuation approaches and metrics: a survey of the theory and evidence}. {[}s.l.{]} Now Publishers Inc, 2007.

  \leavevmode\hypertarget{ref-damodaran2012}{}%
  DAMODARAN, A. \textbf{Investment philosophies: successful strategies and the investors who made them work}. Hoboken, NJ: John Wiley \& Sons, 2012.

  \leavevmode\hypertarget{ref-epe2020}{}%
  EPE. \textbf{Plano Nacional de Energia 2050}. Brasília: Ministério de Minas e Energia, 2020.

  \leavevmode\hypertarget{ref-fama2004}{}%
  FAMA, E. F.; FRENCH, K. R. The capital asset pricing model: Theory and evidence. \textbf{Journal of economic perspectives}, v. 18, n. 3, p. 25--46, 2004.

  \leavevmode\hypertarget{ref-fisher1930}{}%
  FISHER, I. \textbf{Theory of interest: as determined by impatience to spend income and opportunity to invest it}. {[}s.l.{]} Augustusm Kelly Publishers, Clifton, 1930.

  \leavevmode\hypertarget{ref-techreport-exampleIn}{}%
  GARRET, D. A. \textbf{The Microscopic Detection of Corrosion in Aluminum Aircraft Structures with Thermal Neutron Beams and Film Imaging Methods}. Washington, D.C.: National Bureau of Standards, 1977.

  \leavevmode\hypertarget{ref-gordon1959}{}%
  GORDON, M. J. Dividends, earnings, and stock prices. \textbf{The review of economics and statistics}, p. 99--105, 1959.

  \leavevmode\hypertarget{ref-gordon1956}{}%
  GORDON, M. J.; SHAPIRO, E. Capital equipment analysis: the required rate of profit. \textbf{Management science}, v. 3, n. 1, p. 102--110, 1956.

  \leavevmode\hypertarget{ref-graham2016}{}%
  GRAHAM, B. \textbf{O investidor inteligente}. Rio de Janeiro: HarperCollins Brasil, 2016.

  \leavevmode\hypertarget{ref-article-example}{}%
  IESAN, D. Existence Theorems in the Theory of Mixtures. \textbf{Journal of Elasticity}, v. 42, n. 2, p. 145--163, fev. 1996.

  \leavevmode\hypertarget{ref-klayman1995}{}%
  KLAYMAN, J. Varieties of confirmation bias. \textbf{Psychology of learning and motivation}, v. 32, p. 385--418, 1995.

  \leavevmode\hypertarget{ref-kruger1999}{}%
  KRUGER, J.; DUNNING, D. Unskilled and unaware of it: how difficulties in recognizing one's own incompetence lead to inflated self-assessments. \textbf{Journal of personality and social psychology}, v. 77, n. 6, p. 1121, 1999.

  \leavevmode\hypertarget{ref-lie2002}{}%
  LIE, E.; LIE, H. J. Multiples used to estimate corporate value. \textbf{Financial Analysts Journal}, v. 58, n. 2, p. 44--54, 2002.

  \leavevmode\hypertarget{ref-liu2002}{}%
  LIU, J.; NISSIM, D.; THOMAS, J. Equity valuation using multiples. \textbf{Journal of Accounting Research}, v. 40, n. 1, p. 135--172, 2002.

  \leavevmode\hypertarget{ref-liu2007}{}%
  LIU, J.; NISSIM, D.; THOMAS, J. Is cash flow king in valuations? \textbf{Financial Analysts Journal}, v. 63, n. 2, p. 56--68, 2007.

  \leavevmode\hypertarget{ref-markowitz1952}{}%
  MARKOWITZ, H. Portfolio selection. \textbf{The Journal of Finance}, v. 7, n. 1, p. 77--91, 1952.

  \leavevmode\hypertarget{ref-modigliani1958}{}%
  MODIGLIANI, F.; MILLER, M. H. The cost of capital, corporation finance and the theory of investment. \textbf{The American economic review}, v. 48, n. 3, p. 261--297, 1958.

  \leavevmode\hypertarget{ref-munger2006}{}%
  MUNGER, C. T. \textbf{Poor Charlie's Almanack: The Wit and Wisdom of Charles T. Munger}. Virginia Beach: Donning Company, 2006.

  \leavevmode\hypertarget{ref-assafneto2018}{}%
  NETO, A. A. \textbf{Mercado financeiro}. 14. ed. São Paulo: Atlas, 2018.

  \leavevmode\hypertarget{ref-schwarz1991}{}%
  SCHWARZ, N. et al. Ease of retrieval as information: another look at the availability heuristic. \textbf{Journal of Personality and Social psychology}, v. 61, n. 2, p. 195, 1991.

  \leavevmode\hypertarget{ref-sharpe1964}{}%
  SHARPE, W. F. Capital asset prices: A theory of market equilibrium under conditions of risk. \textbf{The journal of finance}, v. 19, n. 3, p. 425--442, 1964.

  \leavevmode\hypertarget{ref-simon1997}{}%
  SIMON, H. A. \textbf{Models of bounded rationality: Empirically grounded economic reason}. Massachusetts: MIT press, 1997. v. 3
  \end{cslreferences}
  \backmatter
  \bibliographystyle{coppe-unsrt}
  \bibliography{thesis}

  %\appendix
  %\include{appenA}
\end{document}
